%% zusammenf.tex
%% $Id: zusammenf.tex 4 2005-10-10 20:51:21Z bless $
%%

\chapter{Zusammenfassung und Ausblick}
\label{ch:Zusammenfassung}
%% ==============================

\begin{itemize}

\item viele Schluessel in Datenbank, aber nur wenige davon nehmen am
  Web of Trust teil.
\item Grosse Mehrzahl der PGP-Benutzer (Leute mit Schluesseln,
  kann allerdings nicht vorrausgesetzt werden, dass diese tatsaechlich
  benutzt werden) authentifiziert Schl\"ussel nicht
\item mesoscopic: MSCC hat ausgepr\"agte modulare Struktur wie viele andere
  soziale Netzwerke
\item Communities stellen Bausteine dar, die das Gesamtnetz ergeben.
\item Einige dieser Communities weisen Charakteristika auf, die eine
  Zuordnung zu einer sozialen Gruppe oder Keysigning-Party erlauben.
\item Aufl\"osungslimit hurts: Zusammenpacken in grosse Communities
  bringt zwar m\"oglicherweise bessere Modularity, erschwert aber die
  inhaltliche Analyse.
\item Allerdings sind diese Mechanismen nicht geeignet, um die
  Entstehungsdynamik insgesamt befriedigend zu erklaeren.
\item Obwohl das nicht untersucht wurde, kann doch vermutet werden,
  dass die \"uberwiegende Mehrheit der Schl\"ussel in der MSCC von
  Personen stammt, die aus der Computersicherheit,  Open-Source-Gemeinde stammen oder einen
  akademischen Background haben. Allerdings gibt es auch eine ganze
  Reihe Communities, die sich einzelnen Firmen zuordnen lassen (sony,
  cisco). 

\end{itemize}

Schl\"ussel und Signaturen enthalten grunds\"atzlich den Zeitpunkt
ihrer Erzeugung und im Keyserver-Netzwerk sind alle jemals
\"offentlich gemachten Schl\"ussel abrufbar. Im Unterschied zu vielen
anderen sozialen Netzwerken ist hier die gesamte
Entwicklungsgeschichte verf\"ugbar. Der in dieser Arbeit berechnete
Datensatz k\"onnte damit ein interessanter Ansatzpunkt sein, um die
Entstehungsdynamik eines komplexen sozialen Netzwerks im Detail zu
untersuchen.

%%% Local Variables: 
%%% mode: latex
%%% TeX-master: "diplarb"
%%% End: 
