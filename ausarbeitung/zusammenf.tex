%% zusammenf.tex
%% $Id: zusammenf.tex 4 2005-10-10 20:51:21Z bless $
%%

\chapter{Zusammenfassung und Ausblick}
\label{ch:Zusammenfassung}
%% ==============================

Es wurde eine Software vorgestellt, die aus der Datenbank eines
Keyservers die Struktur des Zertifikatsgraphen extrahiert. Im
Unterschied zu Wotsap beschränkt sich der Datensatz nicht auf die
größte starke Zusammenhangskomponente, sondern enthält alle
g\"ultigen Schlüssel. Auch die zeitliche Entwicklung ist sichtbar, da
Entstehungs-, Ablauf- und Widerrufszeitpunkte enthalten sind.

Dieser Datensatz wurde benutzt, um die Struktur des Web of Trust auf
mehreren Ebenen zu untersuchen. Es wurde gezeigt, dass die Mehrzahl
der Schlüssel kaum vernetzt ist und sich fast alle Signaturen auf
eine zentrale Komponente von 45.000 Schlüsseln konzentrieren. 

Es wurde argumentiert, dass der Zertifikatsgraph Elemente eines
sozialen Netzwerks aufweist. In der Tat zeigt die zentrale
Komponente Eigenschaften, die typisch für soziale Netzwerke sind:
den Small-World-Effekt, ein hohes Mass an Clustering, eine Korrelation
zwischen dem Grad von Knoten und eine ausgeprägte
Community-Struktur.

Die zentrale Komponente zeigt au{\ss}erdem eine Gradverteilung mit
hoher Variabilit\"at, die einem Power-Law \"ahnelt. Im Unterschied zu
skalenfreien Netzwerken wird der Zusammenhang dieser Komponente aber
nicht fundamental durch wenige stark vernetzte Hubs bestimmt. Das
Netzwerk zeigt sich im Gegenteil recht robust bei einer gezielten
Entfernung der am besten vernetzten Knoten.

Es wurde versucht, die Auswirkung von Fortschritten in Angriffen auf
kryptographische Methoden für das Netzwerk abzuschätzen. Direkt
absehbare Probleme mit MD5, SHA1 und RSA-Schlüsseln geringer Länge
werden den Zusammenhalt des Netzwerks nicht tief greifend beeinflussen,
aber eine Reihe von selbst nicht betroffenen Schlüsseln von diesem
abtrennen.

Die Communities sind nicht zufällig entstanden, sondern scheinen
teilweise tatsächlich soziale Zusammenhänge
widerzuspiegeln. Selbst mit den hier verwendeten primitiven Methoden
und ohne weiteres Wissen über tatsächliche soziale Zusammenhänge
kann für viele Communities gezeigt werden, dass sie aus gemeinsamen
Gruppenzugehörigkeiten entstanden sind. Allerdings kann dadurch
nicht befriedigend erklärt werden, wie die Communities entstehen
und wie sie sich untereinander vernetzen.

Insgesamt hat sich eine überraschend geringe
\emph{öffentliche}\footnote{Es ist nicht bekannt, wie viele Benutzer
  zwar das Web of Trust nutzen, ihre Signaturen aber nicht
  veröffentlichen. Es scheint jedoch nicht wahrscheinlich, dass es
  sich dabei um eine signifikante Zahl handelt.}Nutzung des
PGP-Authentifizerungsmechanismus gezeigt. Nur innerhalb einer Gruppe
von 45000 Schlüsseln sind überhaupt nennenswerte
Signaturaktitiväten feststellbar und auch in dieser sind viele in
Bezug auf Signierungen) vermutlich inaktive Schlüssel enthalten. Die
Anzahl der Personen, die ernsthaft und kontinuierlich zum Web of Trust
beiträgt, ist sehr gering. Die große Mehrheit der Personen, die
ihre Schlüssel veröffentlicht haben, scheint Schlüssel von
Kommunikationspartnern nicht zu authentifizieren.

Auch wenn einige Arbeiten zur Struktur des Web of Trust
veröffentlicht wurden, wurden hier Aspekte untersucht, die dort
keine Beachtung fanden. Außerdem wurde hier der \emph{aktuelle} Stand
des Netzwerks untersucht, dessen Größe sich deutlich verändert
hat.

Im Unterschied zu vielen anderen sozialen Netzwerken ist hier die
gesamte Entwicklungsgeschichte verfügbar. Der in dieser Arbeit
berechnete Datensatz könnte damit ein interessanter Ansatzpunkt
sein, um die Entstehungsdynamik eines komplexen sozialen Netzwerks im
Detail zu untersuchen. Dabei könnte beispielsweise die
Entstehungsdynamik (Entstehung und Verschmelzung) der Communities
verfolgt werden. Auch die Vernetzungsstruktur der Communities in
Berücksichtigung ihrer nationalen Zuordnung und die Rolle einzelner
Knoten innerhalb von Communities (etwa welche Knoten als ``Brücken''
zwischen Communities dienen) könnte untersucht
werden.

%%% Local Variables: 
%%% mode: latex
%%% TeX-master: "diplarb"
%%% End: 
