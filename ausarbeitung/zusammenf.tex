%% zusammenf.tex
%% $Id: zusammenf.tex 4 2005-10-10 20:51:21Z bless $
%%

\chapter{Zusammenfassung und Ausblick}
\label{ch:Zusammenfassung}
%% ==============================

In dieser Arbeit wurden einige Aspekte des OpenPGP Web of Trust
untersucht, wobei der Fokus auf der Struktur des sich daraus
ergebenden Netzwerks lag.
Auch wenn bereits einige Arbeiten zur Struktur des Web of Trust
veröffentlicht wurden, wurden hier Aspekte untersucht, die dort
keine Beachtung fanden. Außerdem wurde hier der \emph{aktuelle} Stand
des Netzwerks untersucht, dessen Größe sich im Vergleich mit dem Stand
der Literatur deutlich verändert hat.

Es wurde eine Software vorgestellt, die aus der Datenbank eines
Keyservers die Struktur des Zertifikatsgraphen extrahiert. Im
Unterschied zu Wotsap beschränkt sich der Datensatz nicht auf die
größte starke Zusammenhangskomponente, sondern enthält alle
g\"ultigen Schlüssel. Auch die zeitliche Entwicklung ist sichtbar, da
Entstehungs-, Ablauf- und Widerrufszeitpunkte enthalten
sind. Zus\"atzlich sind weitere Informationen \"uber die Schl\"ussel
enthalten, die weitergehende Analysen erm\"oglichen.

Dieser Datensatz wurde benutzt, um die Struktur des Web of Trust auf
mehreren Ebenen zu untersuchen. Es wurde gezeigt, dass die Mehrzahl
der vorhandenen Schlüssel kaum vernetzt ist. Auch der Vernetzungsgrad
des Gro{\ss}teils der restlichen Knoten ist so gering, dass die sich
dort ergebenden Komponenten von trivialer Gr\"osse sind. Der
Gro{\ss}teil der vorhandenen Signaturen konzentriert sich auf 
eine zentrale Komponente von 45.000 Schlüsseln. Bei dieser Komponente
scheint es sich also um den einzigen aktiven und nennenwert vernetzten
Teil der Schl\"ussel zu handeln.

Es wurde argumentiert, dass der Zertifikatsgraph Elemente eines
sozialen Netzwerks aufweist. In der Tat zeigt die zentrale
Komponente Eigenschaften, die typisch für soziale Netzwerke sind:
den Small-World-Effekt, ein hohes Mass an Clustering, eine Korrelation
zwischen dem Grad von Knoten und eine ausgeprägte
Community-Struktur. Dadurch wird die Annahme gest\"utzt, dass sich die
\emph{Signaturbeziehungen} auf dem Substrat \emph{sozialer
  Beziehungen} bilden.

Die zentrale Komponente zeigt au{\ss}erdem eine Gradverteilung mit
hoher Variabilit\"at, die einem Power-Law \"ahnelt. Im Unterschied zu
skalenfreien Netzwerken wird der Zusammenhang dieser Komponente aber
nicht fundamental durch wenige stark vernetzte Hubs bestimmt. Das
Netzwerk zeigt sich im Gegenteil recht robust bei einer gezielten
Entfernung der am besten vernetzten Knoten. Sein Zusammenhang scheint
nicht von der Existenz einiger weniger "`Hubs"' abzuh\"angen. Dadurch
wird auch illustriert, dass l\"angst nicht jedes Netzwerk, dessen
Gradverteilung einem Power-Law \"ahnelt, diese Eigenschaft aufweist. Dass sich das Netzwerk bei
der Entfernung zuf\"alliger Knoten wie erwartet sehr robust zeigt,
aber eben auch bei der gezielten Entfernung gut vernetzter Knoten
recht stabil bleibt, ist insbesondere f\"ur seine konkrete Funktion
relevant: Das durchaus nat\"urliche Verschwinden von Knoten aus dem
Netz durch Kompromittierung, Ablaufen etc. scheint seine Funktion auf
den ersten Blick kaum zu beeintr\"achtigen. Ein Angriff auf die
Netzwerkstruktur als Form eines Denial of Service-Angriffs durch die
Entfernung einzelner Schl\"ussel scheint nicht praktikabel.

Es wurde versucht, die Auswirkung von Fortschritten in Angriffen auf
kryptographische Methoden für das Netzwerk abzuschätzen. Direkt
absehbare Probleme mit MD5, SHA1 und RSA-Schlüsseln geringer Länge
werden den Zusammenhalt des Netzwerks nicht fundamental beeinflussen,
aber eine Reihe von selbst nicht betroffenen Schlüsseln von diesem
abtrennen.

Interessant w\"are hier allerdings noch eine Betrachtung der Distanzen
im Netzwerk bei der Entfernung von Schl\"usseln, da eine Erh\"ohung
der durchschnittlichen Distanzen die Benutzbarkeit des Netzwerks
deutlich beschr\"anken k\"onnte. Auch ist nicht klar, ob die Knoten
mit den h\"ochsten Graden tats\"achlich die \emph{zentralsten} Knoten
in Bezug auf Zusammenhalt und Distanzen sind.

Die Communities sind nicht zufällig entstanden, sondern scheinen
"= zumindest teilweise "= tatsächlich soziale Zusammenhänge
widerzuspiegeln. Selbst mit den hier verwendeten einfachen Methoden
und ohne weiteres Wissen über tatsächliche soziale Zusammenhänge
kann für viele Communities gezeigt werden, dass sie aus gemeinsamen
Gruppenzugehörigkeiten entstanden sind. Auch wenn durch soziale
Gruppen und Keysigning-Parties plausibel erkl\"art werden kann, wie
viele Communities "= und damit auch ein Teil der Gesamtstruktur "=
entstanden sind, sind diese Ans\"atze doch nicht ausreichend, um die
Entstehung insgesamt zu erkl\"aren. Wird die Signaturaktivit\"at der
Schl\"ussel betrachtet, scheint es plausible, dass gerade die
Mitglieder klar zuordenbarer Communities eher inaktiv sind, da
anderenfalls die urspr\"ungliche Struktur "`verwischt"' w\"urde.

Insgesamt hat sich eine überraschend geringe
\emph{öffentliche}\footnote{Es ist nicht bekannt, wie viele Benutzer
  zwar das Web of Trust nutzen, ihre Signaturen aber nicht
  veröffentlichen. Es scheint jedoch nicht wahrscheinlich, dass es
  sich dabei um eine signifikante Zahl handelt.}Nutzung des
Authentifizerungsmechanismus Web of Trust gezeigt. Umso detaillierter
das Netzwerk betrachtet wird, umso mehr reduziert sich die Anzahl
vermutlich aktiver Schl\"ussel. W\"ahrend die gro{\ss}e Mehrzahl der
vorhandenen Schl\"ussel \"uberhaupt nicht vernetzt ist, ist auch die
Mehrzahl der Schl\"ussel, die \"uber ein- oder ausgehende Signaturen
verf\"ugen, nur sehr schwach vernetzt. Nur innerhalb einer Gruppe
von 45.000 Schlüsseln ist überhaupt eine nennenswerte
Vernetzungsstruktur feststellbar und auch in dieser sind viele (in
Bezug auf Signierungen) vermutlich inaktive Schlüssel enthalten. Die
Anzahl der Personen, die durch Signaturaktivit\"at ernsthaft zum Web of Trust
beiträgt, scheint in Relation zur Gr\"o{\ss}e der Schl\"sseldatenbank
und in Anbetracht der Tatsache, dass das Web of Trust ein weltweit . Die große Mehrheit der Personen, die
ihre Schlüssel veröffentlicht haben, scheint Schlüssel von
Kommunikationspartnern nicht zu authentifizieren.

Im Unterschied zu vielen anderen sozialen Netzwerken ist f\"ur das Web
of Trust und den in dieser Arbeit vorgestellten Datensatz die
gesamte Entwicklungsgeschichte verfügbar. Der 
Datensatz könnte damit ein interessanter Ansatzpunkt
sein, um die Entstehungsdynamik eines komplexen sozialen Netzwerks im
Detail zu untersuchen. Dabei könnte beispielsweise die
Entstehungsdynamik (Entstehung und Verschmelzung) der Communities
verfolgt werden. Auch die Vernetzungsstruktur der Communities in
Berücksichtigung ihrer nationalen Zuordnung und die Rolle einzelner
Knoten innerhalb von Communities (etwa welche Knoten als "`Brücken"'
zwischen Communities dienen) könnte untersucht
werden.

%%% Local Variables: 
%%% mode: latex
%%% TeX-master: "diplarb"
%%% End: 
