%% zusammenf.tex
%% $Id: zusammenf.tex 4 2005-10-10 20:51:21Z bless $
%%

\chapter{Zusammenfassung und Ausblick}
\label{ch:Zusammenfassung}
%% ==============================

Es wurde eine Software vorgestellt, die aus der Datenbank eines
Keyservers die Struktur des Zertifikatsgraphen extrahiert. Im
Unterschied zu Wotsap beschr\"ankt sich der Datensatz nicht auf die
gr\"osste starke Zusammenhangskomponente, sondern enth\"alt alle
validen Schl\"ussel. Auch die zeitliche Entwicklung ist sichtbar, da
Entstehungs-, Ablauf- und Widerrufszeitpunkte enthalten sind.

Dieser Datensatz wurde benutzt, um die Struktur des Web of Trust auf
mehreren Ebenen zu untersuchen. Es wurde gezeigt, dass die Mehrzahl
der Schl\"ussel kaum vernetzt ist und sich fast alle Signaturen auf
eine zentrale Komponente von 45000 Schl\"usseln konzentrieren. 

Es wurde argumentiert, dass der Zertifikatsgraph Elemente eines
sozialen Netzwerks beinhaltet. In der Tat zeigt die zentrale
Komponente Eigenschaften, die typisch f\"ur soziale Netzwerke sind:
den Small-World-Effekt, ein hohes Mass an Clustering, eine Korrelation
zwischen dem Grad von Knoten und eine ausgepgr\"agte
Community-Struktur.

Die zentrale Komponente zeigt ausserdem eine Gradverteilung, die einem
Power-Law \"ahnelt und eine Eigenschaft, die charakteristisch f\"ur
skalenfreie Netzwerke ist: Eine Struktur von Hubs, die einerseits das
Netzwerk zusammenhalten, es andererseits aber auch verwundbar gegen
gezielte Angriffe machen. 

Es wurde versucht, die Auswirkung von Fortschritten in Angriffen auf
kryptographische Methoden f\"ur das Netzwerk abzusch\"atzen. Direkt
absehbare Probleme mit MD5, SHA1 und RSA-Schl\"usseln geringer L\"ange
werden den Zusammenhalt des Netzwerks nicht tiefgreifend beeinflussen,
aber eine Reihe von selbst nicht betroffenen Schl\"usseln von diesem
abtrennen.

Die Communities sind nicht zuf\"allig entstanden, sondern scheinen
teilweise tats\"achlich soziale Zusammenh\"ange
wiederzuspiegeln. Selbst mit den hier verwendeten primitiven Methoden
und ohne weiteres Wissen \"uber tats\"achliche soziale Zusammenh\"ange
kann f\"ur viele Communities gezeigt werden, dass sie aus gemeinsamen
Gruppenzugeh\"origkeiten entstanden sind. Allerdings kann dadurch
nicht befriedigend erkl\"art werden, wie die Communities entstehen
und wie sie sich untereinander vernetzen.

Insgesamt hat sich eine \"uberraschend geringe
\emph{\"offentliche}\footnote{Es ist nicht bekannt, wie viele Benutzer
  zwar das Web of Trust nutzen, ihre Signaturen aber nicht
  ver\"offentlichen. Es scheint jedoch nicht wahrscheinlich, dass es
  sich dabei um eine signifikante Zahl handelt.}Nutzung des
PGP-Authentifizerungsmechanismus gezeigt. Nur innerhalb einer Gruppe
von 45000 Schl\"usseln sind \"uberhaupt nennenswerte
Signaturaktitiv\"aten feststellbar und auch in dieser sind viele
vermutlich inaktive Schl\"ussel enthalten. Die Anzahl der Personen,
die ernsthaft und kontinuierlich zum Web of Trust beitr\"agt, ist sehr
gering. Die grosse Mehrheit der Personen, die ihre Schl\"ussel
ver\"offentlicht haben, scheint Schl\"ussel von Kommunikationspartnern
nicht zu authentifizieren.

Auch wenn einige Arbeiten zur Struktur des Web of Trust
ver\"offentlicht wurden, wurden hier Aspekte untersucht, die dort
keine Beachtung fanden. Ausserdem wurde hier der \emph{aktuelle} Stand
des Netzwerks untersucht, dessen Gr\"osse sich deutlich ver\"andert
hat.

Im Unterschied zu vielen anderen sozialen Netzwerken ist hier die
gesamte Entwicklungsgeschichte verf\"ugbar. Der in dieser Arbeit
berechnete Datensatz k\"onnte damit ein interessanter Ansatzpunkt
sein, um die Entstehungsdynamik eines komplexen sozialen Netzwerks im
Detail zu untersuchen. Dabei k\"onnte beispielsweise die
Entstehungsdynamik (Enstehung und Verschmelzung) der Communities
verfolgt werden. Auch die Vernetzungsstruktur der Communities in
Ber\"ucksichtigung ihrer nationalen Zuordnung und die Rolle einzelner
Knoten innerhalb von Communities (etwa welche Knoten als ``Br\"ucken''
zwischen Communities dienen) k\"onnte untersucht
werden.

%%% Local Variables: 
%%% mode: latex
%%% TeX-master: "diplarb"
%%% End: 
