\begin{appendix}
  \chapter{Schema der Tabellen}
  \label{sec:schema-der-tabellen}

  FIXME Schema der Tabellen...

  Die Zuordnung einzelner Schlüssel zu ihren starken
  Zusammenhangskomponenten erfolgt über die Tabelle
  \emph{component\_ids}. Diese wird erst in einem separaten Schritt
  befüllt, nachdem die Komponentenstruktur berechnet wurde.

  \begin{figure}[h]
    \centering
    {\scriptsize
      \begin{lstlisting}[language=SQL]
        (SELECT signer, signee
        FROM sigs INNER JOIN keys on sigs.signer = keys.keyid 
        WHERE 
        (keys.revoktime IS NULL OR keys.revoktime > $timestamp) 
        AND (keys.exptime IS NULL OR keys.exptime > $timestamp)
        AND (sigs.revoktime IS NULL OR sigs.revoktime > $timestamp) 
        AND (sigs.exptime IS NULL OR sigs.exptime > $timestamp)) 
        INTERSECT 
        (SELECT signer, signee 
        FROM sigs INNER JOIN keys on sigs.signee = keys.keyid
        WHERE 
        (keys.revoktime IS NULL OR keys.revoktime > $timestamp) 
        AND (keys.exptime IS NULL OR keys.exptime > $timestamp) 
        AND (sigs.revoktime IS NULL OR sigs.revoktime > $timestamp) 
        AND (sigs.exptime IS NULL OR sigs.exptime > $timestamp))"
      \end{lstlisting}
    }
    \caption{Abfrage aller zum Zeitpunkt \$timestamp gültigen Signaturen}
    \label{fig:all-valid-keys}
  \end{figure}


  Als Beispiel gibt Abb. \ref{fig:all-valid-keys} ein SQL-Statement an,
  mit dem alle gültigen (d.h. nicht zurückgezogenen oder abgelaufenen)
  Signaturen zu einem bestimmten Zeitpunkt abgerufen werden können.

\chapter{Analysewerkzeuge}
\label{cha:analysewerkzeuge}


\begin{table}

  \begin{tabular}[h]{|l|p{9cm}|}
    \hline
    basic-properties-mpi & Verteilung von Eingangs- und Ausgangsgraden,
    Komponentengrössen, Nachbarschaften, Durchmesser, Radius,
    durschnittliche Pfadlängen (parallelisiert mittels
    MPI\\
    \hline
    betweeness-mpi & Berechnung der Betweeness-Zentralität
    (parallelisiert mittels MPI) \\
    \hline
    clustering-coefficient-mpi & Berechnung des Clustering Coefficient
    (parallelisiert mittels MPI) \\
    \hline
    export & Export des Graphen in verschiedene Dateiformate (igraph,
    Cytoscape) \\
    \hline
    meta-graph & Zeichnung der Struktur der starken
    Zusammenhangskomponenten \\
    \hline
    db-scc-information & Befüllt die SQL-Datenbank mit der Zuordnung
    von Knoten zu Zusammenhangskomponenten \\
    \hline
    dump-sql & Legt die extrahierten Daten einmalig in der
    SQL-Datenbank ab \\
    \hline 
    investigate-component & Untersucht einzelne
    Zusammenhangskomponente in Bezug auf Herkunft der UserIDs
    (Domains) und den Zeitraum der Entstehung der Signaturen \\
    \hline
    mscc-size & Entwicklung der Schlüsselanzahl der grössten starken
    Zusammenhangskomponente \\
    \hline
    simple-stats & Statistiken über die Verwendung bestimmter
    Schlüsseleigenschaften (Algorithmen, Schlüssellängen usw.) \\
    time & Entwicklung der Verwendung von Signatur- und
    Public-Key-Algorithmen und deren Schlüssellängen\\
    \hline
  \end{tabular}
  \caption{Foobar}
  \label{tab:tools}
\end{table}
\end{appendix}
