\begin{appendix}
  \chapter{Schema der Tabellen}
  \label{sec:schema-der-tabellen}


  \begin{table}[ht!]
    \centering
    \begin{tabular}{|c|c|c|c|c|c|c|c|}
      \hline
      keyid & version & puid & ctime & exptime & revoktime & alg &
      keylen \\
      \hline
    \end{tabular}
    \caption{Tabelle \emph{keys}}
    \label{tab:keys}
  \end{table}

  \begin{table}[ht!]
    \centering
    \begin{tabular}{|c|c|}
      \hline
      keyid & uids \\
      \hline
    \end{tabular}
    \caption{Tabelle \emph{uids}}
    \label{tab:uids}
  \end{table}

  \begin{table}[ht!]
    \centering
    \begin{tabular}{|c|c|c|c|c|c|c|c|}
      \hline
      signer & signee & level & exptime & ctime & revoktime & hash\_alg
      & pk\_alg \\
      \hline
    \end{tabular}
    \caption{Tabelle \emph{sigs}}
    \label{tab:sigs}
  \end{table}

  Die Zuordnung einzelner Schlüssel zu ihren starken
  Zusammenhangskomponenten erfolgt über die Tabelle
  \emph{component\_ids}. Diese wird erst in einem separaten Schritt
  befüllt, nachdem die Komponentenstruktur berechnet wurde.

  \begin{table}[ht!]
    \centering
    \begin{tabular}{|c|c|}
      \hline
      keyid & component\_id \\
      \hline
    \end{tabular}
    \caption{Tabelle \emph{component\_ids}}
    \label{tab:component_ids}
  \end{table}

  \begin{figure}[h]
    \centering
    {\scriptsize
      \begin{lstlisting}[language=SQL]
        (SELECT signer, signee
        FROM sigs INNER JOIN keys on sigs.signer = keys.keyid 
        WHERE 
        (keys.revoktime IS NULL OR keys.revoktime > $timestamp) 
        AND (keys.exptime IS NULL OR keys.exptime > $timestamp)
        AND (sigs.revoktime IS NULL OR sigs.revoktime > $timestamp) 
        AND (sigs.exptime IS NULL OR sigs.exptime > $timestamp)) 
        INTERSECT 
        (SELECT signer, signee 
        FROM sigs INNER JOIN keys on sigs.signee = keys.keyid
        WHERE 
        (keys.revoktime IS NULL OR keys.revoktime > $timestamp) 
        AND (keys.exptime IS NULL OR keys.exptime > $timestamp) 
        AND (sigs.revoktime IS NULL OR sigs.revoktime > $timestamp) 
        AND (sigs.exptime IS NULL OR sigs.exptime > $timestamp))
      \end{lstlisting}
    }
    \caption{Abfrage aller zum Zeitpunkt \$timestamp gültigen Signaturen}
    \label{fig:all-valid-keys}
  \end{figure}

  Als Beispiel gibt Abb. \ref{fig:all-valid-keys} ein SQL-Statement an,
  mit dem alle gültigen (d.h. nicht zurückgezogenen oder abgelaufenen)
  Signaturen zu einem bestimmten Zeitpunkt abgerufen werden können.

\chapter{Analysewerkzeuge}
\label{cha:analysewerkzeuge}

Die folgende Liste beschreibt die Aufteilung der Analyseaufgaben auf
die einzelnen Programme.

\begin{description}
\item[basic\_properties\_mpi] Verteilung von Eingangs- und
  Ausgangsgraden, Komponentengrössen, Nachbarschaften, Durchmesser,
  Radius, durschnittliche Pfadlängen (parallelisiert mittels MPI)
\item[correlate\_deg] Korrelation von ein- und ausgehendem Grad
\item[without] Gr\"ossenentwicklung des Netzwerks, wenn Knoten nach
  bestimmten Kriterien (Grad, zuf\"allig) entfernt werden
\item[time] Zeitliche Entwicklungen (Anzahl Knoten etc.)
\item[simple\_stats] Statistiken \"uber Benutzung von Algorithmen etc.
\item[mscc\_size] Zeitliche Entwicklung der Gr\"osse der MSCC
\item[clustering\_coefficient\_ser] Berechnung des
  Clustering-Koeffizienten
\item[betweeness\_mpi] Berechnung der Betweeness-Centrality
  (parallelisiert mittels MPI
\item[investigate\_component] Untersucht einzelne
  Zusammenhangskomponente in Bezug auf Herkunft der UserIDs
  (Domains) und den Zeitraum der Entstehung der Signaturen 
\item[investigate\_communities] Untersucht Communities in Bezug auf
  Herkunft der UserIDs (Domains) und den Zeitraum der Entstehung der
  Signaturen
\item[component\_structure] Struktur der starken
  Zusammenhangskomponenten
\item[community\_structure] Struktur der Communities
\item[export] Export des Graphen in verschiedene Dateiformate (igraph,
  Cytoscape)
\item[db-scc-information] Befüllt die SQL-Datenbank mit der
  Zuordnung von Knoten zu Zusammenhangskomponenten
\end{description}
\end{appendix}
