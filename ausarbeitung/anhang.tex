\begin{appendix}
  \chapter{Schema der Tabellen}
  \label{sec:schema-der-tabellen}

  FIXME Schema der Tabellen...

  Die Zuordnung einzelner Schlüssel zu ihren starken
  Zusammenhangskomponenten erfolgt über die Tabelle
  \emph{component\_ids}. Diese wird erst in einem separaten Schritt
  befüllt, nachdem die Komponentenstruktur berechnet wurde.

  \begin{figure}[h]
    \centering
    {\scriptsize
      \begin{lstlisting}[language=SQL]
        (SELECT signer, signee
        FROM sigs INNER JOIN keys on sigs.signer = keys.keyid 
        WHERE 
        (keys.revoktime IS NULL OR keys.revoktime > $timestamp) 
        AND (keys.exptime IS NULL OR keys.exptime > $timestamp)
        AND (sigs.revoktime IS NULL OR sigs.revoktime > $timestamp) 
        AND (sigs.exptime IS NULL OR sigs.exptime > $timestamp)) 
        INTERSECT 
        (SELECT signer, signee 
        FROM sigs INNER JOIN keys on sigs.signee = keys.keyid
        WHERE 
        (keys.revoktime IS NULL OR keys.revoktime > $timestamp) 
        AND (keys.exptime IS NULL OR keys.exptime > $timestamp) 
        AND (sigs.revoktime IS NULL OR sigs.revoktime > $timestamp) 
        AND (sigs.exptime IS NULL OR sigs.exptime > $timestamp))"
      \end{lstlisting}
    }
    \caption{Abfrage aller zum Zeitpunkt \$timestamp gültigen Signaturen}
    \label{fig:all-valid-keys}
  \end{figure}

  Als Beispiel gibt Abb. \ref{fig:all-valid-keys} ein SQL-Statement an,
  mit dem alle gültigen (d.h. nicht zurückgezogenen oder abgelaufenen)
  Signaturen zu einem bestimmten Zeitpunkt abgerufen werden können.
\end{appendix}
