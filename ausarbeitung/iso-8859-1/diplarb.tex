\documentclass{wissdoc}
% Autor: Roland Bless 1996-2009, bless <at> kit.edu
% ----------------------------------------------------------------
% Diplomarbeit - Hauptdokument
% ----------------------------------------------------------------
%%
%% $Id: diplarb.tex 53 2009-12-10 12:23:37Z bless $
%%
% wissdoc Optionen: draft, relaxed, pdf --> siehe wissdoc.cls
% ------------------------------------------------------------------
% Weitere packages: (Dokumentation dazu durch "latex <package>.dtx")
\usepackage{bibgerm}
\usepackage[numbers,sort&compress]{natbib}
% \usepackage{varioref}
% \usepackage{verbatim}
% \usepackage{float}    %z.B. \floatstyle{ruled}\restylefloat{figure}
% \usepackage{subfigure}
% \usepackage{fancybox} % f�r schattierte,ovale Boxen etc.
% \usepackage{tabularx} % automatische Spaltenbreite
% \usepackage{supertab} % mehrseitige Tabellen
% \usepackage[svnon,svnfoot]{svnver} % SVN Versionsinformation 
%% ---------------- end of usepackages -------------

%\svnversion{$Id: diplarb.tex 53 2009-12-10 12:23:37Z bless $} % In case that you want to include version information in the footer

%% Informationen f�r die PDF-Datei
\hypersetup{
 pdfauthor={N.N.},
 pdftitle={Not set}
 pdfsubject={Not set},
 pdfkeywords={Not set}
}

% Macros, nicht unbedingt notwendig
%%%%%%%%%%%%%%%%%%%%%%%%%%%%%%%%%%%%%%%%%%%%%%%%%%%%%%%%%%
% macros.tex -- einige mehr oder weniger nuetzliche Makros
% Autor: Roland Bless 1998
%%%%%%%%%%%%%%%%%%%%%%%%%%%%%%%%%%%%%%%%%%%%%%%%%%%%%%%%%%
% $Id: macros.tex 33 2007-01-23 09:00:59Z bless $
%%%%%%%%%%%%%%%%%%%%%%%%%%%%%%%%%%%%%%%%%%%%%%%%%%%%%%%%%%


%%%%%%%%%%%%%%%%%%%%%%%
% Kommentare 
%%%%%%%%%%%%%%%%%%%%%%%
\ifnotdraftelse{
\newcommand{\Kommentar}[1]{}
}{\newcommand{\Kommentar}[1]{{\em #1}}}
% Alles innerhalb von \Hide{} oder \ignore{} 
% wird von LaTeX komplett ignoriert (wie ein Kommentar)
\newcommand{\Hide}[1]{}
\let\ignore\Hide

%%%%%%%%%%%%%%%%%%%%%%%%%
% Leere Seite ohne Seitennummer, wird aber gezaehlt
%%%%%%%%%%%%%%%%%%%%%%%%%

\newcommand{\leereseite}{% Leerseite ohne Seitennummer, n�chste Seite rechts (wenn 2-seitig)
 \clearpage{\pagestyle{empty}\cleardoublepage}
}
%%%%%%%%%%%%%%%%%%%%%%%%%%
% Flattersatz rechts und Silbentrennung, Leerraum nach rechts maximal 1cm
%%%%%%%%%%%%%%%%%%%%%%%%%%
\makeatletter
\newcommand{\myraggedright}{%
 \let\\\@centercr\@rightskip 0pt plus 1cm
 \rightskip\@rightskip
  \leftskip\z@skip
  \parindent\z@
  \spaceskip=.3333em
  \xspaceskip=.5em}
\makeatother

\makeatletter
\newcommand{\mynewline}{%
 \@centercr\@rightskip 0pt plus 1cm
}
\makeatother


%%%%%%%%%%%%%%%%%%%%%%%%%%
% F�r Index
%%%%%%%%%%%%%%%%%%%%%%%%%%
\makeatletter
\def\mydotfill{\leavevmode\xleaders\hb@xt@ .44em{\hss.\hss}\hfill\kern\z@}
\makeatother
\def\bold#1{{\bfseries #1}}
\newbox\dbox \setbox\dbox=\hbox to .4em{\hss.\hss} % dot box for leaders
\newskip\rrskipb \rrskipb=.5em plus3em % ragged right space before break
\newskip\rrskipa \rrskipa=-.17em plus -3em minus.11em % ditto, after
\newskip\rlskipa \rlskipa=0pt plus3em % ragged left space after break
\newskip\rlskipb \rlskipb=.33em plus-3em minus.11em % ragged left before break
\newskip\lskip \lskip=3.3\wd\dbox plus1fil minus.3\wd\dbox % for leaders
\newskip \lskipa \lskipa=-2.67em plus -3em minus.11em %after leaders
\mathchardef\rlpen=1000 \mathchardef\leadpen=600
\def\rrspace{\nobreak\hskip\rrskipb\penalty0\hskip\rrskipa}
\def\rlspace{\penalty\rlpen\hskip\rlskipb\vadjust{}\nobreak\hskip\rlskipa}
\let\indexbreak\rlspace
\def\raggedurl{\penalty10000 \hskip.5em plus15em \penalty0 \hskip-.17em plus-15em minus.11em}
\def\raggeditems{\nobreak\hskip\rrskipb \penalty\leadpen \hskip\rrskipa %
\vadjust{}\nobreak\leaders\copy\dbox\hskip\lskip %
\kern3em \penalty\leadpen \hskip\lskipa %
\vadjust{}\nobreak\hskip\rlskipa}
\renewcommand*\see[2]{\rlspace\emph{\seename}~#1} % from makeidx.sty

%%%%%%%%%%%%%%%%%%%%%%%%%%
% Neue Seite rechts, leere linke Seite ohne Headings
%%%%%%%%%%%%%%%%%%%%%%%%%%
\newcommand{\xcleardoublepage}
{{\pagestyle{empty}\cleardoublepage}}

%%%%%%%%%%%%%%%%%%%%%%%%%%
% Tabellenspaltentypen (benoetigt colortbl)
%%%%%%%%%%%%%%%%%%%%%%%%%%
\newcommand{\PBS}[1]{\let\temp=\\#1\let\\=\temp}
\newcolumntype{y}{>{\PBS{\raggedright\hspace{0pt}}}p{1.35cm}}
\newcolumntype{z}{>{\PBS{\raggedright\hspace{0pt}}}p{2.5cm}}
\newcolumntype{q}{>{\PBS{\raggedright\hspace{0pt}}}p{6.5cm}}
\newcolumntype{g}{>{\columncolor[gray]{0.8}}c} % Grau
\newcolumntype{G}{>{\columncolor[gray]{0.9}}c} % helleres Grau

%%%%%%%%%%%%%%%%%%%%%%%%%%
% Anf�hrungszeichen oben und unten
%%%%%%%%%%%%%%%%%%%%%%%%%%
\newcommand{\anf}[1]{"`{#1}"'}

%%%%%%%%%%%%%%%%%%%%%%%%%%
% Tiefstellen von Text
%%%%%%%%%%%%%%%%%%%%%%%%%%
% S\tl{0} setzt die 0 unter das S (ohne Mathemodus!)
% zum Hochstellen gibt es uebrigens \textsuperscript
\makeatletter
\DeclareRobustCommand*\textlowerscript[1]{%
  \@textlowerscript{\selectfont#1}}
\def\@textlowerscript#1{%
  {\m@th\ensuremath{_{\mbox{\fontsize\sf@size\z@#1}}}}}
\let\tl\textlowerscript
\let\ts\textsuperscript
\makeatother

%%%%%%%%%%%%%%%%%%%%%%%%%%
% Gau�-Klammern
%%%%%%%%%%%%%%%%%%%%%%%%%%
\newcommand{\ceil}[1]{\lceil{#1}\rceil}
\newcommand{\floor}[1]{\lfloor{#1}\rfloor}

%%%%%%%%%%%%%%%%%%%%%%%%%%
% Average Operator (analog zu min, max)
%%%%%%%%%%%%%%%%%%%%%%%%%%
\def\avg{\mathop{\mathgroup\symoperators avg}}

%%%%%%%%%%%%%%%%%%%%%%%%%%
% Wortabk�rzungen
%%%%%%%%%%%%%%%%%%%%%%%%%%
\def\zB{z.\,B.\ }
\def\dh{d.\,h.\ }
\def\ua{u.\,a.\ }
\def\su{s.\,u.\ }
\newcommand{\bzw}{bzw.\ }

%%%%%%%%%%%%%%%%%%%%%%%%%%%%%%%%%%%
% Einbinden von Graphiken
%%%%%%%%%%%%%%%%%%%%%%%%%%%%%%%%%%%
% global scaling factor
\def\gsf{0.9}
%% Graphik, 
%% 3 Argumente: Datei, Label, Unterschrift
\newcommand{\Abbildung}[3]{%
\begin{figure}[tbh] %
\centerline{\scalebox{\gsf}{\includegraphics*{#1}}} %
\caption{#3} %
\label{#2} %
\end{figure} %
}
\let\Abb\Abbildung
%% Abbps
%% Graphik, skaliert, Angabe der Position
%% 5 Argumente: Position, Breite (0 bis 1.0), Datei, Label, Unterschrift
\newcommand{\Abbildungps}[5]{%
\begin{figure}[#1]%
\begin{center}
\scalebox{\gsf}{\includegraphics*[width=#2\textwidth]{#3}}%
\caption{#5}%
\label{#4}%
\end{center}
\end{figure}%
}
\let\Abbps\Abbildungps
%% Graphik, Angabe der Position, frei w�hlbares Argument f�r includegraphics
%% 5 Argumente: Position, Optionen, Datei, Label, Unterschrift
\newcommand{\Abbildungpf}[5]{%
\begin{figure}[#1]%
\begin{center}
\scalebox{\gsf}{\includegraphics*[#2]{#3}}%
\caption{#5}%
\label{#4}%
\end{center}
\end{figure}%
}
\let\Abbpf\Abbildungpf

%%
% Anmerkung: \resizebox{x}{y}{box} skaliert die box auf Breite x und H�he y,
%            ist x oder y ein !, dann wird das uspr�ngliche 
%            Seitenverh�ltnis beibehalten.
%            \rescalebox funktioniert �hnlich, nur das dort ein Faktor
%            statt einer Dimension angegeben wird.
%%
% \Abbps{Position}{Breite in Bruchteilen der Textbreite}{Dateiname}{Label}{Bildunterschrift}
%

\newcommand{\refAbb}[1]{%
s.~Abbildung \ref{#1}}

%%%%%%%%%%%%%%%%%%%%
%% end of macros.tex
%%%%%%%%%%%%%%%%%%%%

% Print URLs not in Typewriter Font
\def\UrlFont{\rm}

\newcommand{\blankpage}{% Leerseite ohne Seitennummer, n�chste Seite rechts
 \clearpage{\pagestyle{empty}\cleardoublepage}
}

%% Einstellungen f�r das gesamte Dokument

% Trennhilfen
% Wichtig! 
% Im german-paket sind zus�tzlich folgende Trennhinweise enthalten:
% "- = zus�tzliche Trennstelle
% "| = Vermeidung von Ligaturen und m�gliche Trennung (bsp: Schaf"|fell)
% "~ = Bindestrich an dem keine Trennung erlaubt ist (bsp: bergauf und "~ab)
% "= = Bindestrich bei dem Worte vor und dahinter getrennt werden d�rfen
% "" = Trennstelle ohne Erzeugung eines Trennstrichs (bsp: und/""oder)

% Trennhinweise fuer Woerter hier beschreiben
\hyphenation{
% Pro-to-koll-in-stan-zen
% Ma-na-ge-ment  Netz-werk-ele-men-ten
% Netz-werk Netz-werk-re-ser-vie-rung
% Netz-werk-adap-ter Fein-ju-stier-ung
% Da-ten-strom-spe-zi-fi-ka-tion Pa-ket-rumpf
% Kon-troll-in-stanz
}

% Index-Datei �ffnen
\ifnotdraft{\makeindex}
%%%%%%%%%%%%%% includeonly %%%%%%%%%%%%%%%%%%%
% Es werden nur die Teile eingebunden, die hier 
% aufgefuehrt sind!
\includeonly{%
titelseite,%
erklaerung,% Ist in KA Pflicht f�r Diplomarbeiten
einleitung,% Motivation, Zielsetzung, Gliederung
grundlagen,% Grundlagen 
analyse,   % Problembeschreibung (Detail) und Related Work
entwurf,   % Beschreibung der Probleml�sung (Konzepte, allg. Architektur, ...)
implemen,  % Beschreibung der Umsetzung/Implementierung
eval,      % Nachweis und Auswertung
zusammenf  % Zusammenfassung der Ergebnisse und Ausblick
}
%%%%%%%%%%%%%%%%%%%%%%%%%%%%%%%%%%%%%%%%%%%%%%
\begin{document}

\frontmatter
\pagenumbering{roman}
\ifnotdraft{
 %% Titelseite
%% Vorlage $Id: titelseite.tex 54 2009-12-10 12:23:58Z bless $

\def\usesf{}
\let\usesf\sffamily % diese Zeile auskommentieren für normalen TeX Font

\newsavebox{\Erstgutachter}
\savebox{\Erstgutachter}{\usesf Prof.~Dr.~Peter~Hauck}

\begin{titlepage}
\setlength{\unitlength}{1pt}

\thispagestyle{empty}

%\begin{titlepage}
%%\let\footnotesize\small \let\footnoterule\relax
\begin{center}
\hbox{}
\vfill
{\usesf
{\huge\bfseries Analyse des OpenPGP Web of Trust \par}
\vskip 1.8cm
Studienarbeit\\
von\\[2mm]
\vskip 1cm

{\large\bfseries Alexander Ulrich\\}
\vskip 1.2cm
Lehrstuhl für Rechnernetze und Internet\\
Wilhelm-Schickard-Institut für Informatik\\
Fakultät für Informations- und Kognitionswissenschaften\\
Universität Tübingen
%Universität Karlsruhe (TH)\\[2ex]
\vskip 3cm
\begin{tabular}{p{5.5cm}l}
Erstgutachter: & \usebox{\Erstgutachter} \\
Betreuender~Mitarbeiter: & Dipl.-Inform.~Ralph~Holz \\
\end{tabular}
\vskip 3cm
Bearbeitungszeit:\qquad 08.~Monat~2009 -- 01.~Monat~2010
}
\end{center}
\vfill
\end{titlepage}
%% Titelseite Ende


%%% Local Variables: 
%%% mode: latex
%%% TeX-master: "diplarb"
%%% End: 

 \blankpage % Leerseite auf Titelr�ckseite
 %
 % Die folgende Erkl�rung ist f�r Diplomarbeiten Pflicht
 % (siehe Pr�fungsordnung), f�r Studienarbeiten nicht notwendig
 \thispagestyle{empty}
\vspace*{42\baselineskip}
\hbox to \textwidth{\hrulefill}
\par
Ich erkl�re hiermit, dass ich die vorliegende Arbeit selbst�ndig verfasst und
keine anderen als die angegebenen Quellen und Hilfsmittel verwendet habe.

Karlsruhe, den ??. ?????? 200?

%%%%%%%%%%%%%%%%%%%%%%%%%%%%%%%%%%%%%%%%%%%%%%%%%%%%%%%%%%%%%%%%%%%%%%%%
%% Hinweis:
%%
%% Diese Erkl�rung wird von der Pr�fungsordnung f�r Diplomarbeiten 
%% verlangt und ist zu unterschreiben. F�r Studienarbeiten ist diese
%% Erkl�rung nicht zwingend notwendig, schadet aber auch nicht.
%%%%%%%%%%%%%%%%%%%%%%%%%%%%%%%%%%%%%%%%%%%%%%%%%%%%%%%%%%%%%%%%%%%%%%%%
\clearpage







 \blankpage % Leerseite auf Erkl�rungsr�ckseite
}
%
%% *************** Hier geht's ab ****************
%% ++++++++++++++++++++++++++++++++++++++++++
%% Verzeichnisse
%% ++++++++++++++++++++++++++++++++++++++++++
\ifnotdraft{
{\parskip 0pt\tableofcontents} % toc bitte einzeilig
\blankpage
%\listoffigures
%\blankpage
%\listoftables
%\blankpage
}


%% ++++++++++++++++++++++++++++++++++++++++++
%% Hauptteil
%% ++++++++++++++++++++++++++++++++++++++++++
\graphicspath{{Bilder/}}

\mainmatter
\pagenumbering{arabic}
%% Einleitung.tex
%% $Id: einleitung.tex 28 2007-01-18 16:31:32Z bless $
%%

\chapter{Einleitung}
\label{ch:Einleitung}
%% ==============================

Die meisten Kommunikationsmedien -- insbesondere E-Mail -- können
nicht garantieren, dass Nachrichten unterwegs abgehört, verändert
oder mit gefälschtem Absender verschickt werden. PGP und GnuPG sind
Softwarepakete, die -- basierend auf asymmetrischer Kryptographie --
Vertraulichkeit und Authentizität gewährleisten.

Asymmetrische Kryptographie erlaubt es, Schlüssel auf unsicheren
Kanälen auszutauschen. Damit stellt sich aber ein neues Problem: Wie
kann ein Benutzer sicherstellen, dass ein Schlüssel authentisch ist,
also tatsächlich vom angeblichen Besitzer stammt? Im Kontext
elektronischer Kommunikation kann dies meist nicht durch direkten
Kontakt mit dem Kommunikationspartner festgestellt werden. Statt
dessen müssen sich Teilnehmer auf Dritte verlassen, die Schlüssel
verifizieren.

Der traditionelle Ansatz dafür besteht in der Verwendung von
zentralen und hierarchischen Infrastrukturen, die sich auf wenige
zentrale Autoritäten verlassen. Diese Autoritäten nehmen die
Aufgabe der Verifizierung von Schlüsseln wahr.  PGP setzt
stattdessen auf das Paradigma der Selbstorganisierung in Form des
\emph{Web of Trust}. Mit Selbstorganisierung ist gemeint, dass ein
System nicht zentral geplant wird, sondern sich ausschließlich aus
den Interaktionen aller Teilnehmer ergibt. Im Web of Trust ist die
Aufgabe der Verifizierung von Schlüsseln auf \emph{alle} Teilnehmer
verlagert. Diese stellen Informationen über von ihnen verifizierte
Schlüssel öffentlich zur Verfügung und ermöglichen es so
anderen Teilnehmern, sie zu benutzen. Auf diese Weise ergibt sich ein
Netzwerk, das abbildet, welche Schlüssel (bzw. Benutzer) welche
anderen Schlüssel verifiziert haben.

Eine Vielzahl von Arbeiten befasst sich mit der Analyse der Struktur
großer Netzwerke, die in verschiedensten Bereichen angetroffen
werden. Dies können etwa soziale, technische oder biologische
Netzwerke sein. In den letzten Jahren hat insbesondere die Analyse von
sozialen und technischen Netzwerken durch die Verfügbarkeit von
Datensammlungen aus dem Bereich des Internets, von Online Social
Networks (Facebook etc.) u.ä. einen erheblichen Auftrieb erhalten.

Das Web of Trust stellt ein Beispiel für ein solches Netzwerk dar,
das sich aus der Selbstorganisierung der Teilnehmer ergibt. Auch wenn
es zunächst nicht über die streng regelmäßige Struktur eines
geplanten Systems verfügt, ist es auch nicht komplett
unstrukturiert. Es ergeben sich Strukturen, die die Beziehungen der
Teilnehmer im realen Leben widerspiegeln. Das Netzwerk kann
einerseits nur in Bezug auf die Erfüllung seines Zwecks betrachtet
werden: Wie stark wird das Web of Trust benutzt und wie gut erfüllt
es seinen Zweck? Andererseits kann seine Struktur benutzt werden, um
die Mechanismen seiner Entstehung zu erklären. Wie finden sich die
Mechanismen seiner Entstehung und die abgebildeten Beziehungen in
dieser Struktur wieder? Diese Studienarbeit soll zur Beantwortung beider Fragen beitragen.


%% ==============================
\section{Zielsetzung der Arbeit}
%% ==============================
\label{ch:Einleitung:sec:Zielsetzung}

Das Ziel dieser Studienarbeit ist eine Analyse der Struktur des Web of
Trust auf verschiedenen Ebenen. Dazu soll zunächst eine Software
implementiert werden, die das Web of Trust aus einer
Schlüsseldatenbank extrahiert und in Form eines Graphen verfügbar
macht. 

Der dadurch erhaltene Graph soll unter mehreren Gesichtspunkten
analysiert werden:

\begin{itemize}

\item Wie stark sind die Teilnehmer des Web of Trust vernetzt?

\item Bekannt ist aus der Literatur, dass sich im Web of Trust eine starke
  Zusammenhangskomponente findet, die deutlich gr\"o{\ss}er ist als
  die restlichen Komponenten. Allerdings enth\"alt diese trotzdem nur
  einen kleinen Teil der insgesamt vorhandenen Schl\"ussel. Hier soll
  auch die Struktur des restlichen Netzwerks betrachtet werden.

\item  Wie nützlich ist das Web of Trust für seine Teilnehmer? Als
  Mass für die Nützlichkeit wird hier angesehen, wie viele
  Schlüssel für Teilnehmer anhand der Signaturen im Netzwerk
  prinzipiell verifizierbar sind.

\item Wie robust ist das Web of Trust? Wie sehr hängt sein
  Zusammenhang von einzelnen Schlüsseln ab?

\item Wie häufig werden verschiedene Verschlüsselungs- und
  Hashalgorithmen bei PGP-Schlüsseln verwendet? Welche Auswirkungen
  haben Fortschritte in Angriffen gegen diese Algorithmen?

\item Spiegeln sich Gruppierungen von Individuen in der Struktur des
  Web of Trust wieder? Lässt sich anhand solcher Gruppen erklären,
  wie sich das Netzwerk bildet?
\end{itemize}

%% ==============================
\section{Gliederung der Arbeit}
%% ==============================
\label{ch:Einleitung:sec:Gliederung}

Die Arbeit ist im weiteren wie folgt gegliedert: Kapitel
\ref{ch:Grundlagen} erläutert die notwendigen Grundlagen. Dies sind
zum einen die Prinzipien asymmetrischer Kryptographie (Abschnitt
\ref{ch:Grundlagen:sec:PublicKeyCrypto}), Methoden zur
Authentifizierung öffentlicher Schlüssel, die Weiterhin werden in
Abschnitt \ref{ch:Grundlagen:sec:PGP} die Prinzipien von PGP und GnuPG
sowie die Rahmenbedingungen, in denen PGP/GnuPGP verwendet wird,
beschrieben. Außerdem werden in Abschnitt \ref{sec:graph-und-netzw}
die graphentheoretischen und netzwerkanalytischen Verfahren
eingeführt, die in dieser Arbeit verwendet werden. Abschnitt
\ref{ch:Grundlagen:sec:RelatedWork} bespricht verwandte Arbeiten.

Kapitel \ref{ch:Methoden} beschreibt die verwendeten
Methoden. Abschnitt \ref{ch:Grundlagen:sec:WarumEigene} begründet
zunächst, warum die Extraktion der benötigen Daten selbst
implementiert wurde. Abschnitt \ref{ch:Grundlagen:sec:Design} beschreibt die
im Rahmen dieser Arbeit implementierte Software zur Extraktion und
Analyse des Web of Trust, während Abschnitt
\ref{sec:community-analyse} die Vorgehensweise für die Analyse der
Community-Struktur in Abschnitt \ref{sec:result-zusamm-und-comm}
erläutert.

Kapitel \ref{ch:Ergebnisse} stellt die Ergebnisse vor und
diskutiert sie. Dies gliedert sich in Ergebnisse über die Struktur
des Web of Trust (Abschnitt \ref{sec:result-allg-merkm-des}), einige
Statistiken über Eigenschaften einzelner Schlüssel und die
zeitliche Entwicklung des Web of Trust (Abschnitt
\ref{sec:result-key-properties}) sowie die Analyse der
Community-Struktur (Abschnitt \ref{sec:result-zusamm-und-comm}).

Kapitel \ref{ch:Zusammenfassung} schließlich fasst die Ergebnisse
zusammen und bietet einen Ausblick auf Ansatzpunkte für zukünftige
Arbeiten.

%%% Local Variables: 
%%% mode: latex
%%% TeX-master: "diplarb"
%%% End: 
  % Einleitung
%% grundlagen.tex
%% $Id: grundlagen.tex 28 2007-01-18 16:31:32Z bless $
%%

\chapter{Grundlagen}
\label{ch:Grundlagen}
%% ==============================
Die Grundlagen müssen soweit beschrieben
werden, dass ein Leser das Problem und
die Problemlösung  versteht.Um nicht zuviel 
zu beschreiben, kann man das auch erst gegen 
Ende der Arbeit schreiben.

Bla fasel\ldots

%% ==============================
\section{Kryptographie mit öffentlichen Schlüsseln}
%% ==============================
\label{ch:Grundlagen:sec:PublicKeyCrypto}

\subsection{Prinzip}
\label{ch:Grundlagen:sec:PublicKeyCrypto:subsec:Prinzip}

\subsection{Authentisierung von Schlüsseln}
\label{ch:Grundlagen:sec:PublicKeyCrypto:subsec:KeyAuth}

\subsubsection{Zentrale PKI}
\label{ch:Grundlagen:sec:PublicKeyCrypto:subsec:KeyAuth:subsubsec:PKI}

\subsubsection{Web of Trust}
\label{ch:Grundlagen:sec:PublicKeyCrypto:subsec:KeyAuth:subsubsec:WOT}

%% ==============================
\section{OpenPGP (RFC4880)}
%% ==============================
\label{ch:Grundlagen:sec:OpenPGP}

\subsection{Paketformat v4}
\label{ch:Grundlagen:sec:OpenPGP:subsec:PaketFormat}

\subsection{Unterschiede v3}
\label{ch:Grundlagen:sec:OpenPGP:subsec:v3Format}

\section{Graphentheorie allgemein}
\label{ch:Grundlagen:sec:Graphentheorie}

\section{Netzwerkanalyse}
\label{ch:Grundlagen:sec:Netzwerkanalyse}

\subsection{Netzwerkstatistiken}
\label{ch:Grundlagen:sec:Netzwerkanalyse:subsec:Statistiken}

\subsection{Communities}
\label{ch:Grundlagen:sec:Netzwerkanalyse:subsec:Communities}





%% ==============================
\section{Verwandte Arbeiten}
%% ==============================
\label{ch:Grundlagen:sec:RelatedWork}



\subsection{Analysen des OpenPGP-Web of Trust}
\label{ch:Grundlagen:sec:RelatedWork:subsec:wot-analysis}

\subsection{Community-Strukturen allgemein}
\label{ch:Grundlagen:sec:RelatedWork:subsec:community-analysis}



%%% Local Variables: 
%%% mode: latex
%%% TeX-master: "diplarb"
%%% End: 
  % Grundlagen
%% analyse.tex
%% $Id: analyse.tex 28 2007-01-18 16:31:32Z bless $

\chapter{Analyse}
\label{ch:Analyse}
%% ==============================
In diesem Kapitel sollten zunächst das zu lösende Problem
sowie die Anforderungen und die Randbedingungen 
einer Lösung\index{Lösung} beschrieben werden (also nochmal
eine präzisierte Aufgabenstellung\index{Aufgabenstellung}).

Dann folgt üblicherweise ein Überblick über bereits existierende
Lösungen bzw. Ansätze, die meistens andere Voraussetzungen bzw.
Randbedingungen annehmen.

Bla fasel\ldots

%% ==============================
\section{Anforderungen}
%% ==============================
\label{ch:Analyse:sec:Anforderungen}
Anforderungen und Randbedingungen\index{Randbedingungen} \ldots

%% ==============================
\section{Existierende Lösungsansätze}
%% ==============================
\label{ch:Analyse:sec:RelatedWork}

Hier kommt eine ausführliche Diskussion
von "`Related Work"'.

Bla fasel\ldots

%% ==============================
\section{Weiterer Abschnitt}
%% ==============================
\label{ch:Analyse:sec:Abschnitt}

Bla fasel\ldots hat auch schon \cite{TB2000} gesagt und
\cite{TB98,JSAC96,qosr} sollte man mal gelesen haben.
Abbildung~\ref{fig:test} auf S.~\pageref{fig:test} sollte man
sich mal anschauen.

Blindtext Blindtext Blindtext Blindtext Blindtext Blindtext Blindtext
Blindtext Blindtext Blindtext Blindtext Blindtext Blindtext Blindtext
Blindtext Blindtext Blindtext Blindtext Blindtext Blindtext Blindtext
Blindtext Blindtext Blindtext Blindtext Blindtext Blindtext Blindtext
Blindtext Blindtext Blindtext Blindtext Blindtext Blindtext Blindtext
Blindtext Blindtext Blindtext Blindtext Blindtext Blindtext Blindtext
Blindtext Blindtext Blindtext Blindtext Blindtext Blindtext Blindtext
Blindtext Blindtext Blindtext Blindtext Blindtext Blindtext Blindtext
Blindtext Blindtext Blindtext Blindtext Blindtext Blindtext Blindtext

Blindtext Blindtext Blindtext Blindtext Blindtext Blindtext Blindtext
Blindtext Blindtext Blindtext Blindtext Blindtext Blindtext Blindtext
Blindtext Blindtext Blindtext Blindtext Blindtext Blindtext Blindtext
Blindtext Blindtext Blindtext Blindtext Blindtext Blindtext Blindtext
Blindtext Blindtext Blindtext Blindtext Blindtext Blindtext Blindtext
Blindtext Blindtext Blindtext Blindtext Blindtext Blindtext Blindtext
Blindtext Blindtext Blindtext Blindtext Blindtext Blindtext Blindtext
Blindtext Blindtext Blindtext Blindtext Blindtext Blindtext Blindtext
Blindtext Blindtext Blindtext Blindtext Blindtext Blindtext Blindtext
Blindtext Blindtext Blindtext Blindtext Blindtext Blindtext Blindtext
Blindtext Blindtext Blindtext Blindtext Blindtext Blindtext Blindtext
Blindtext Blindtext Blindtext Blindtext Blindtext Blindtext Blindtext
Blindtext Blindtext Blindtext Blindtext Blindtext Blindtext Blindtext

Blindtext Blindtext Blindtext Blindtext Blindtext Blindtext Blindtext
Blindtext Blindtext Blindtext Blindtext Blindtext Blindtext Blindtext
Blindtext Blindtext Blindtext Blindtext Blindtext Blindtext Blindtext
Blindtext Blindtext Blindtext Blindtext Blindtext Blindtext Blindtext
Blindtext Blindtext Blindtext Blindtext Blindtext Blindtext Blindtext
Blindtext Blindtext Blindtext Blindtext Blindtext Blindtext Blindtext
Blindtext Blindtext Blindtext Blindtext Blindtext Blindtext Blindtext
Blindtext Blindtext Blindtext Blindtext Blindtext Blindtext Blindtext
Blindtext Blindtext Blindtext Blindtext Blindtext Blindtext Blindtext
Blindtext Blindtext Blindtext Blindtext Blindtext Blindtext Blindtext

Blindtext Blindtext Blindtext Blindtext Blindtext Blindtext Blindtext
Blindtext Blindtext Blindtext Blindtext Blindtext Blindtext Blindtext
Blindtext Blindtext Blindtext Blindtext Blindtext Blindtext Blindtext
Blindtext Blindtext Blindtext Blindtext Blindtext Blindtext Blindtext
Blindtext Blindtext Blindtext Blindtext Blindtext Blindtext Blindtext
Blindtext Blindtext Blindtext Blindtext Blindtext Blindtext Blindtext
Blindtext Blindtext Blindtext Blindtext Blindtext Blindtext Blindtext
Blindtext Blindtext Blindtext\index{Blindtext} Blindtext Blindtext Blindtext Blindtext

\begin{figure}[!htbp]
  \centering
  \fbox{\parbox{0.8\textwidth}{
  Abbildungen sollten möglichst als EPS (Encapsulated Postscript) 
  bzw. PDF eingebunden werden.
  Zur Erzeugung sauberer EPS-Dateien empfiehlt sich das Tool \texttt{ps2eps}
  zur Nachbearbeitung von Postscript-Dateien. Mit \texttt{epstopdf} kann
  dann eine PDF-Datei zum Einbinden erzeugt werden.}}
  \caption{Testabbildung}
  \label{fig:test}
\end{figure}

Blindtext Blindtext Blindtext Blindtext Blindtext Blindtext Blindtext
Blindtext Blindtext Blindtext Blindtext Blindtext Blindtext Blindtext
Blindtext Blindtext Blindtext Blindtext Blindtext Blindtext Blindtext
Blindtext Blindtext Blindtext Blindtext Blindtext Blindtext Blindtext
Blindtext Blindtext Blindtext Blindtext Blindtext Blindtext Blindtext
Blindtext Blindtext Blindtext Blindtext Blindtext Blindtext Blindtext
Blindtext Blindtext Blindtext Blindtext Blindtext Blindtext Blindtext

Blindtext Blindtext Blindtext Blindtext Blindtext Blindtext Blindtext
Blindtext Blindtext Blindtext Blindtext Blindtext Blindtext Blindtext
Blindtext Blindtext Blindtext Blindtext Blindtext Blindtext Blindtext

Blindtext Blindtext Blindtext Blindtext Blindtext Blindtext Blindtext
Blindtext Blindtext Blindtext Blindtext Blindtext Blindtext Blindtext
Blindtext Blindtext Blindtext Blindtext Blindtext Blindtext Blindtext
Blindtext Blindtext Blindtext Blindtext Blindtext Blindtext Blindtext
Blindtext Blindtext Blindtext Blindtext Blindtext Blindtext Blindtext
Blindtext Blindtext Blindtext Blindtext Blindtext Blindtext Blindtext

Blindtext Blindtext Blindtext Blindtext Blindtext Blindtext Blindtext
Blindtext Blindtext Blindtext Blindtext Blindtext Blindtext Blindtext
Blindtext Blindtext Blindtext Blindtext Blindtext Blindtext Blindtext
Blindtext Blindtext Blindtext Blindtext Blindtext Blindtext Blindtext
Blindtext Blindtext Blindtext Blindtext Blindtext Blindtext Blindtext
Blindtext Blindtext Blindtext Blindtext Blindtext Blindtext Blindtext
Blindtext Blindtext Blindtext Blindtext Blindtext Blindtext Blindtext
Blindtext Blindtext Blindtext Blindtext Blindtext Blindtext Blindtext
Blindtext Blindtext Blindtext Blindtext Blindtext Blindtext Blindtext
Blindtext Blindtext Blindtext Blindtext Blindtext Blindtext Blindtext
Blindtext Blindtext Blindtext Blindtext Blindtext Blindtext Blindtext
Blindtext Blindtext Blindtext Blindtext Blindtext Blindtext Blindtext
Blindtext Blindtext Blindtext Blindtext Blindtext Blindtext Blindtext
Blindtext Blindtext Blindtext Blindtext Blindtext Blindtext Blindtext

Blindtext Blindtext Blindtext Blindtext Blindtext Blindtext Blindtext
Blindtext Blindtext Blindtext Blindtext Blindtext Blindtext Blindtext
Blindtext Blindtext Blindtext Blindtext Blindtext Blindtext Blindtext
Blindtext Blindtext Blindtext Blindtext Blindtext Blindtext Blindtext
Blindtext Blindtext Blindtext Blindtext Blindtext Blindtext Blindtext
Blindtext Blindtext Blindtext Blindtext Blindtext Blindtext Blindtext
Blindtext Blindtext Blindtext Blindtext Blindtext Blindtext Blindtext
Blindtext Blindtext Blindtext Blindtext Blindtext Blindtext Blindtext
Blindtext Blindtext Blindtext Blindtext Blindtext Blindtext Blindtext
Blindtext Blindtext Blindtext Blindtext Blindtext Blindtext Blindtext
Blindtext Blindtext Blindtext Blindtext Blindtext Blindtext Blindtext
%% ==============================
\section{Zusammenfassung}
%% ==============================
\label{ch:Analyse:sec:zusammenfassung}

Am Ende sollten ggf. die wichtigsten Ergebnisse nochmal in \emph{einem}
kurzen Absatz zusammengefasst werden.

%%% Local Variables: 
%%% mode: latex
%%% TeX-master: "diplarb"
%%% End: 
     % Analyse
%% entwurf.tex
%% $Id: entwurf.tex 28 2007-01-18 16:31:32Z bless $
%%

\chapter{Entwurf}
\label{ch:Entwurf}
%% ==============================
In diesem Kapitel erfolgt die ausf�hrliche Beschreibung des eigenen
L�sungsansatzes. Dabei sollten L�sungsalternativen diskutiert und
Entwurfsentscheidungen dargelegt werden.


Bla fasel\ldots

%% ==============================
\section{Abschnitt 1}
%% ==============================
\label{ch:Entwurf:sec:Abschnitt1}

Bla fasel\ldots

%% ==============================
\section{Abschnitt 2}
%% ==============================
\label{ch:Entwurf:sec:Abschnitt2}

Bla fasel\ldots

Blindtext Blindtext Blindtext Blindtext Blindtext Blindtext Blindtext
Blindtext Blindtext Blindtext Blindtext Blindtext Blindtext Blindtext
Blindtext Blindtext Blindtext Blindtext Blindtext Blindtext Blindtext
Blindtext Blindtext Blindtext Blindtext Blindtext Blindtext Blindtext
Blindtext Blindtext Blindtext Blindtext Blindtext Blindtext Blindtext
Blindtext Blindtext Blindtext Blindtext Blindtext Blindtext Blindtext
Blindtext Blindtext Blindtext Blindtext Blindtext Blindtext Blindtext
Blindtext Blindtext Blindtext Blindtext Blindtext Blindtext Blindtext
Blindtext Blindtext Blindtext Blindtext Blindtext Blindtext Blindtext
Blindtext Blindtext Blindtext Blindtext Blindtext Blindtext Blindtext
Blindtext Blindtext Blindtext Blindtext Blindtext Blindtext Blindtext
Blindtext Blindtext Blindtext Blindtext Blindtext Blindtext Blindtext
Blindtext Blindtext Blindtext Blindtext Blindtext Blindtext Blindtext
Blindtext Blindtext Blindtext Blindtext Blindtext Blindtext Blindtext
Blindtext Blindtext Blindtext Blindtext Blindtext Blindtext Blindtext
Blindtext Blindtext Blindtext Blindtext Blindtext Blindtext Blindtext
Blindtext Blindtext Blindtext Blindtext Blindtext Blindtext Blindtext

Blindtext Blindtext Blindtext Blindtext Blindtext Blindtext Blindtext
Blindtext Blindtext Blindtext Blindtext Blindtext Blindtext Blindtext
Blindtext Blindtext Blindtext Blindtext Blindtext Blindtext Blindtext
Blindtext Blindtext Blindtext Blindtext Blindtext Blindtext Blindtext
Blindtext Blindtext Blindtext Blindtext Blindtext Blindtext Blindtext
Blindtext Blindtext Blindtext Blindtext Blindtext Blindtext Blindtext
Blindtext Blindtext Blindtext Blindtext Blindtext Blindtext Blindtext
Blindtext Blindtext Blindtext Blindtext Blindtext Blindtext Blindtext
Blindtext Blindtext Blindtext Blindtext Blindtext Blindtext Blindtext
Blindtext Blindtext Blindtext Blindtext Blindtext Blindtext Blindtext
Blindtext Blindtext Blindtext Blindtext Blindtext Blindtext Blindtext
Blindtext Blindtext Blindtext Blindtext Blindtext Blindtext Blindtext
Blindtext Blindtext Blindtext Blindtext Blindtext Blindtext Blindtext
Blindtext Blindtext Blindtext Blindtext Blindtext Blindtext Blindtext
Blindtext Blindtext Blindtext Blindtext Blindtext Blindtext Blindtext
Blindtext Blindtext Blindtext Blindtext Blindtext Blindtext Blindtext
Blindtext Blindtext Blindtext Blindtext Blindtext Blindtext Blindtext
Blindtext Blindtext Blindtext Blindtext Blindtext Blindtext Blindtext
Blindtext Blindtext Blindtext Blindtext Blindtext Blindtext Blindtext
Blindtext Blindtext Blindtext Blindtext Blindtext Blindtext Blindtext

Blindtext Blindtext Blindtext Blindtext Blindtext Blindtext Blindtext
Blindtext Blindtext Blindtext Blindtext Blindtext Blindtext Blindtext
Blindtext Blindtext Blindtext Blindtext Blindtext Blindtext Blindtext
Blindtext Blindtext Blindtext Blindtext Blindtext Blindtext Blindtext
Blindtext Blindtext Blindtext Blindtext Blindtext Blindtext Blindtext
Blindtext Blindtext Blindtext Blindtext Blindtext Blindtext Blindtext
Blindtext Blindtext Blindtext Blindtext Blindtext Blindtext Blindtext
Blindtext Blindtext Blindtext Blindtext Blindtext Blindtext Blindtext
Blindtext Blindtext Blindtext Blindtext Blindtext Blindtext Blindtext
Blindtext Blindtext Blindtext Blindtext Blindtext Blindtext Blindtext
Blindtext Blindtext Blindtext Blindtext Blindtext Blindtext Blindtext
Blindtext Blindtext Blindtext Blindtext Blindtext Blindtext Blindtext
Blindtext Blindtext Blindtext Blindtext Blindtext Blindtext Blindtext
Blindtext Blindtext Blindtext Blindtext Blindtext Blindtext Blindtext
Blindtext Blindtext Blindtext Blindtext Blindtext Blindtext Blindtext

Blindtext Blindtext Blindtext Blindtext Blindtext Blindtext Blindtext
Blindtext Blindtext Blindtext Blindtext Blindtext Blindtext Blindtext
Blindtext Blindtext Blindtext Blindtext Blindtext Blindtext Blindtext
Blindtext Blindtext Blindtext Blindtext Blindtext Blindtext Blindtext
Blindtext Blindtext Blindtext Blindtext Blindtext Blindtext Blindtext
Blindtext Blindtext Blindtext Blindtext Blindtext Blindtext Blindtext
Blindtext Blindtext Blindtext Blindtext Blindtext Blindtext Blindtext
Blindtext Blindtext Blindtext Blindtext Blindtext Blindtext Blindtext
Blindtext Blindtext Blindtext Blindtext Blindtext Blindtext Blindtext
Blindtext Blindtext Blindtext Blindtext Blindtext Blindtext Blindtext
Blindtext Blindtext Blindtext Blindtext Blindtext Blindtext Blindtext
Blindtext Blindtext Blindtext Blindtext Blindtext Blindtext Blindtext
Blindtext Blindtext Blindtext Blindtext Blindtext Blindtext Blindtext
Blindtext Blindtext Blindtext Blindtext Blindtext Blindtext Blindtext
Blindtext Blindtext Blindtext Blindtext Blindtext Blindtext Blindtext
Blindtext Blindtext Blindtext Blindtext Blindtext Blindtext Blindtext

Blindtext Blindtext Blindtext Blindtext Blindtext Blindtext Blindtext
Blindtext Blindtext Blindtext Blindtext Blindtext Blindtext Blindtext
Blindtext Blindtext Blindtext Blindtext Blindtext Blindtext Blindtext
Blindtext Blindtext Blindtext Blindtext Blindtext Blindtext Blindtext
Blindtext Blindtext Blindtext Blindtext Blindtext Blindtext Blindtext
Blindtext Blindtext Blindtext Blindtext Blindtext Blindtext Blindtext
Blindtext Blindtext Blindtext Blindtext Blindtext Blindtext Blindtext
Blindtext Blindtext Blindtext Blindtext Blindtext Blindtext Blindtext
Blindtext Blindtext Blindtext Blindtext Blindtext Blindtext Blindtext
Blindtext Blindtext Blindtext Blindtext Blindtext Blindtext Blindtext
Blindtext Blindtext Blindtext Blindtext Blindtext Blindtext Blindtext
Blindtext Blindtext Blindtext Blindtext Blindtext Blindtext Blindtext
Blindtext Blindtext Blindtext Blindtext Blindtext Blindtext Blindtext

%% ==============================
\section{Zusammenfassung}
%% ==============================
\label{ch:Entwurf:sec:zusammenfassung}

Am Ende sollten ggf. die wichtigsten Ergebnisse nochmal in \emph{einem}
kurzen Absatz zusammengefasst werden.

%%% Local Variables: 
%%% mode: latex
%%% TeX-master: "diplarb"
%%% End: 
     % Entwurf
%% implemen.tex
%% $Id: implemen.tex 4 2005-10-10 20:51:21Z bless $
%%

\chapter{Implementierung}
\label{ch:Implementierung}
%% ==============================
Bla fasel\ldots

%% ==============================
\section{Abschnitt 1}
%% ==============================
\label{ch:Implementierung:sec:Abschnitt1}

Bla fasel\ldots

%% ==============================
\section{Abschnitt 2}
%% ==============================
\label{ch:Implementierung:sec:Abschnitt2}

Bla fasel\ldots

%%% Local Variables: 
%%% mode: latex
%%% TeX-master: "diplarb"
%%% End: 
    % Implementierung
%% eval.tex
%% $Id: eval.tex 5 2005-10-10 20:55:48Z bless $

\chapter{Evaluierung}
\label{ch:Evaluierung}
%% ==============================
Hier kommt der Nachweis, dass das in Kapitel~\ref{ch:Entwurf}
entworfene Konzept auch funktioniert. Leistungsmessungen einer
Implementierung werden auch immer gerne gesehen.

Bla fasel\ldots

%% ==============================
\section{Abschnitt 1}
%% ==============================
\label{ch:Evaluierung:sec:Abschnitt1}

Bla fasel\ldots

%% ==============================
\section{Abschnitt 2}
%% ==============================
\label{ch:Evaluierung:sec:Abschnitt2}

Bla fasel\ldots

%% ==============================
\section{Zusammenfassung}
%% ==============================
\label{ch:Evaluierung:sec:zusammenfassung}

Am Ende sollten ggf. die wichtigsten Ergebnisse nochmal in \emph{einem}
kurzen Absatz zusammengefasst werden.

%%% Local Variables: 
%%% mode: latex
%%% TeX-master: "diplarb"
%%% End: 
        % Evaluierung
%% zusammenf.tex
%% $Id: zusammenf.tex 4 2005-10-10 20:51:21Z bless $
%%

\chapter{Zusammenfassung und Ausblick}
\label{ch:Zusammenfassung}
%% ==============================

Es wurde eine Software vorgestellt, die aus der Datenbank eines
Keyservers die Struktur des Zertifikatsgraphen extrahiert. Im
Unterschied zu Wotsap beschränkt sich der Datensatz nicht auf die
größte starke Zusammenhangskomponente, sondern enthält alle
g\"ultigen Schlüssel. Auch die zeitliche Entwicklung ist sichtbar, da
Entstehungs-, Ablauf- und Widerrufszeitpunkte enthalten sind.

Dieser Datensatz wurde benutzt, um die Struktur des Web of Trust auf
mehreren Ebenen zu untersuchen. Es wurde gezeigt, dass die Mehrzahl
der Schlüssel kaum vernetzt ist und sich fast alle Signaturen auf
eine zentrale Komponente von 45.000 Schlüsseln konzentrieren. 

Es wurde argumentiert, dass der Zertifikatsgraph Elemente eines
sozialen Netzwerks beinhaltet. In der Tat zeigt die zentrale
Komponente Eigenschaften, die typisch für soziale Netzwerke sind:
den Small-World-Effekt, ein hohes Mass an Clustering, eine Korrelation
zwischen dem Grad von Knoten und eine ausgeprägte
Community-Struktur.

Die zentrale Komponente zeigt außerdem eine Gradverteilung, die einem
Power-Law ähnelt und eine Eigenschaft, die charakteristisch für
skalenfreie Netzwerke ist: Eine Struktur von Hubs, die einerseits das
Netzwerk zusammenhalten, es andererseits aber auch verwundbar gegen
gezielte Angriffe machen. 

Die zentrale Komponente zeigt au{\ss}erdem eine Gradverteilung mit
hoher Variabilit\"at, die einem Power-Law \"ahnelt. Im Unterschied zu
skalenfreien Netzwerken wird der Zusammenhang dieser Komponente aber
nicht fundamental durch wenige stark vernetzte Hubs bestimmt. Das
Netzwerk zeigt sich im Gegenteil recht robust bei einer gezielten
Entfernung der am besten vernetzten Knoten.

Es wurde versucht, die Auswirkung von Fortschritten in Angriffen auf
kryptographische Methoden für das Netzwerk abzuschätzen. Direkt
absehbare Probleme mit MD5, SHA1 und RSA-Schlüsseln geringer Länge
werden den Zusammenhalt des Netzwerks nicht tief greifend beeinflussen,
aber eine Reihe von selbst nicht betroffenen Schlüsseln von diesem
abtrennen.

Die Communities sind nicht zufällig entstanden, sondern scheinen
teilweise tatsächlich soziale Zusammenhänge
widerzuspiegeln. Selbst mit den hier verwendeten primitiven Methoden
und ohne weiteres Wissen über tatsächliche soziale Zusammenhänge
kann für viele Communities gezeigt werden, dass sie aus gemeinsamen
Gruppenzugehörigkeiten entstanden sind. Allerdings kann dadurch
nicht befriedigend erklärt werden, wie die Communities entstehen
und wie sie sich untereinander vernetzen.

Insgesamt hat sich eine überraschend geringe
\emph{öffentliche}\footnote{Es ist nicht bekannt, wie viele Benutzer
  zwar das Web of Trust nutzen, ihre Signaturen aber nicht
  veröffentlichen. Es scheint jedoch nicht wahrscheinlich, dass es
  sich dabei um eine signifikante Zahl handelt.}Nutzung des
PGP-Authentifizerungsmechanismus gezeigt. Nur innerhalb einer Gruppe
von 45000 Schlüsseln sind überhaupt nennenswerte
Signaturaktitiväten feststellbar und auch in dieser sind viele in
Bezug auf Signierungen) vermutlich inaktive Schlüssel enthalten. Die
Anzahl der Personen, die ernsthaft und kontinuierlich zum Web of Trust
beiträgt, ist sehr gering. Die große Mehrheit der Personen, die
ihre Schlüssel veröffentlicht haben, scheint Schlüssel von
Kommunikationspartnern nicht zu authentifizieren.

Auch wenn einige Arbeiten zur Struktur des Web of Trust
veröffentlicht wurden, wurden hier Aspekte untersucht, die dort
keine Beachtung fanden. Außerdem wurde hier der \emph{aktuelle} Stand
des Netzwerks untersucht, dessen Größe sich deutlich verändert
hat.

Im Unterschied zu vielen anderen sozialen Netzwerken ist hier die
gesamte Entwicklungsgeschichte verfügbar. Der in dieser Arbeit
berechnete Datensatz könnte damit ein interessanter Ansatzpunkt
sein, um die Entstehungsdynamik eines komplexen sozialen Netzwerks im
Detail zu untersuchen. Dabei könnte beispielsweise die
Entstehungsdynamik (Entstehung und Verschmelzung) der Communities
verfolgt werden. Auch die Vernetzungsstruktur der Communities in
Berücksichtigung ihrer nationalen Zuordnung und die Rolle einzelner
Knoten innerhalb von Communities (etwa welche Knoten als ``Brücken''
zwischen Communities dienen) könnte untersucht
werden.

%%% Local Variables: 
%%% mode: latex
%%% TeX-master: "diplarb"
%%% End: 
   % Zusammenfassung und Ausblick

%% ++++++++++++++++++++++++++++++++++++++++++
%% Anhang
%% ++++++++++++++++++++++++++++++++++++++++++

\appendix
%\include{anhang_a}
%\include{anhang_b}

%% ++++++++++++++++++++++++++++++++++++++++++
%% Literatur
%% ++++++++++++++++++++++++++++++++++++++++++
%  mit dem Befehl \nocite werden auch nicht 
%  zitierte Referenzen abgedruckt
\cleardoublepage
\phantomsection
\addcontentsline{toc}{chapter}{\bibname}
%%
\nocite{*} % nur angeben, wenn auch nicht im Text zitierte Quellen 
           % erscheinen sollen
\bibliographystyle{itmabbrv} % mit abgek�rzten Vornamen der Autoren
%\bibliographystyle{gerplain} % abbrvnat unsrtnat
% spezielle Zitierstile: Labels mit vier Buchstaben und Jahreszahl
%\bibliographystyle{itmalpha}  % ausgeschriebene Vornamen der Autoren
\bibliography{diplarb}
%% ++++++++++++++++++++++++++++++++++++++++++
%% Index
%% ++++++++++++++++++++++++++++++++++++++++++
\ifnotdraft{
\cleardoublepage
\phantomsection
\printindex            % Index, Stichwortverzeichnis
}
\end{document}
%% end of file
