%% grundlagen.tex
%% $Id: grundlagen.tex 28 2007-01-18 16:31:32Z bless $
%%

\chapter{Grundlagen}
\label{ch:Grundlagen}
%% ==============================

%% ==============================
\section{Kryptographie mit öffentlichen Schlüsseln}
%% ==============================
\label{ch:Grundlagen:sec:PublicKeyCrypto}

\subsection{Prinzip}
\label{ch:Grundlagen:sec:PublicKeyCrypto:subsec:Prinzip}

Klassische Kryptographie basiert auf einem \emph{Geheimnis} oder
\emph{privaten Schl\"ussel}, der beiden Kommunikationspartnern bekannt
ist. 

\subsection{Authentisierung von Schlüsseln}
\label{ch:Grundlagen:sec:PublicKeyCrypto:subsec:KeyAuth}

Zweck von PKI oder Web of Trust ist es, die Authentizit\"at der
Bindung von Schl\"ussel/Zertifikat und Identit\"at zu etablieren. Und
das eben auch dann, wenn man selbst diese Bindung nicht \"uberpr\"ufen
kann.

\subsubsection{Zentrale PKI}
\label{ch:Grundlagen:sec:PublicKeyCrypto:subsec:KeyAuth:subsubsec:PKI}

\subsubsection{Web of Trust}
\label{ch:Grundlagen:sec:PublicKeyCrypto:subsec:KeyAuth:subsubsec:WOT}

\section{PGP/GnuPG}
\label{ch:Grundlagen:sec:PGP}

\subsection{Geschichte von PGP und GnuPG}
\label{ch:Grundlagen:sec:PGP:subsec:Geschichte}

\subsection{Eigenschaften/Fähigkeiten der Implementierungen}
\label{ch:Grundlagen:sec:PGP:subsec:Eigenschaften}

\subsection{Das GnuPG-Vertrauensmodell}
\label{sec:das-gnupg-vertrauensmodell}

Öffentliche PGP-Schlüssel werden oft nicht in einer Weise übergeben,
die die sofortige Verifikation des Schlüssels zulässt, beispielsweise
bei einem persönlichen Treffen. Stattdessen werden Schlüssel häufig
per E-Mail, über einen Keyserver oder andere elektronische Wege
ausgetauscht.  Überprüfung der Authentizität eines Schlüssels ist
deswegen von grosser Bedeutung.

Ein Schlüssel wird von GnuPG genau dann als gültig (\emph{valid})
betrachtet, wenn er die folgenden Bedingungen erfüllt:

\begin{enumerate}
\item Der Schlüssel wurde von ausreichend vielen \emph{gültigen} Schlüsseln
  unterschrieben, d.h. er wurde mindestens entweder von
  \begin{itemize}
  \item dem Besitzer des Schlüsselrings selbst (d.h. von einem
    Schlüssel mit \emph{ultimate trust}) unterschrieben
  \item mindestens $N$ gültigen Schlüsseln, denen voll vertraut wird, unterschrieben
  \item mindestens $M$ gültigen Schlüsseln, denen geringfügig
    vertraut wird, unterschrieben
  \end{itemize}
\item Eine Signaturkette wird nur verwendet, wenn sie ausgehend vom
  Besitzer des Schlüsselrings maximal die Länge $L$ hat.
\end{enumerate}

Ein Schlüssel, der von weniger voll bzw. geringfügig
vertrauenswürdigen Schlüsseln als notwendig unterschrieben wurde, wird
als eingeschränkt gültig (\emph{marginally valid})
angesehen. Allerdings werden Schlüssel dieser Kategorie von GnuPG
genauso wie nicht gültige Schlüssel behandelt.

GnuPG verwendet in der Standardeinstellung die Werte $N=1$, $M=3$ und
$L=5$. Damit wird \emph{ein} Zertifikat, dass von einem Schl\"ussel
mit vollem Vertrauen ausgestellt wurde, als ausreichend
betrachtet. F\"ur Schl\"ussel, denen nur geringf\"ugig vertraut wird,
ist ein einzelnes Zertifikat nicht ausreichend. Dieses muss noch durch
2 weitere solche Zertifikate best\"atigt werden. 

GnuPG erlaubt es einem Anwender, die Parameter $N, M$ und $L$ selbst
zu setzen und damit seine Sicherheitsanforderungen umzusetzen. Je
höher beispielsweise die notwendige Anzahl von Signaturen, um so
kleiner ist der Schaden, den eine einzelne fehlerhafte Signatur
anrichten kann. Ist die maximale Pfadlänge auf einen kleinen Wert
begrenzt, so ist auch die maximale Anzahl der Signaturen auf dem Pfad
kleiner, die potentiell fehlerhaft sein können. Andererseits
verringert sich damit die Anzahl der Signaturen (Pfade im Web of
Trust), die für die Verifizierung benutzt werden können, und damit die
Anzahl verifizierbarer Schlüssel. Es muss also eine Abwägung zwischen
dem Sicherheitsbedürfnis des Nutzers und der praktischen Benutzbarkeit
getroffen werden.

\begin{figure}[t]
  \centering
  \includegraphics[scale=1.0]{images/trust-beispiel.pdf}
  \caption{Beispiel für die Berechnung der Gültigkeit von Schlüsseln}
  \label{fig:trust-beispiel}
\end{figure}

Abbildung \ref{fig:trust-beispiel} gibt ein Beispiel für die
Berechnung der Schlüsselgültigkeit unter Verwendung der
Standard-Parameter $N=1$, $M=3$ und $L=5$. Die Schlüssel $B, H$ und
$K$ wurden direkt von $A$, dem Besitzer des Schlüsselrings,
unterschrieben und sind deshalb voll gültig. Da $L, M,$ und $N$ von
$K$ unterschrieben wurden und dieser über volles Vertrauen verfügt,
sind sie ebenfalls voll gültig. $O$ sowie $J$ sind voll gültig, da sie
jeweils von drei Schlüsseln mit geringfügigem Vertrauen unterschrieben
wurden. $I$ und $P$ wurden dagegen jeweils nur von zwei Schlüsseln mit
geringfügigem Vertrauen unterschrieben und sind deshalb nicht voll
sondern nur eingeschränkt gültig. $G$ wurde zwar von einem voll
gültigen Schlüssel mit vollem Vertrauen unterschrieben. Allerdings
überschreitet die Signaturkette zu $A$ die maximale Länge von 5 und
wird deshalb nicht betrachtet.

Ein öffentlicher Schlüssel, der anhand dieser Regeln nicht als
authentisch verifiziert werden kann, kann trotzdem zur Verschlüsselung
und zur Verifizierung von Signaturen verwendet werden. Allerdings
warnt GnuPG in diesem Fall vor der Verwendung.

\subsection{Soziale Komponente des Web of Trust}
\label{sec:sozi-komp-des}

PGP-CAs (CACert?, CT) erwaehnen, Crypto-Kampagne der CT

In einer Reihe von Open-Source-Projekten spielen PGP und das Web of
Trust eine wichtige Rolle. Im Debian-Projekt beispielweise werden
PGP-Schl\"ussel unter anderem benutzt, um alle bereitgestellten
Softwarepakete durch den zust\"andigen Entwickler signieren zu
lassen. Dazu muss jeder Debian-Entwickler \"uber einen Schl\"ussel
verf\"ugen, der von mindestens einem anderen Debian-Entwickler
verifiziert und unterschrieben wurde. Auf diese Weise bildet sich
bereits innerhalb des Projekts ein Web of Trust. PGP scheint eine
wichtige Rolle in der Kultur des Projekts zu spielen. 



%% ==============================
\section{Der OpenPGP-Standard (RFC4880)}
%% ==============================
\label{ch:Grundlagen:sec:OpenPGP}

\subsection{Paketformat v4}
\label{ch:Grundlagen:sec:OpenPGP:subsec:PaketFormat}

\subsection{Unterschiede v3}
\label{ch:Grundlagen:sec:OpenPGP:subsec:v3Format}

In einem schlichten Graphen haben Kreise mindestens 3 verschiedene
Knoten. Kreise der L\"ange 1 werden durch Schleifen, Kreise der
L\"ange 2 durch Mehrfachkanten gebildet.

\section{Graphentheorie und Netzwerkanalyse}
\label{sec:graph-und-netzw}

\subsection{Graphentheorie allgemein}
\label{ch:Grundlagen:sec:Graphentheorie}

Dieser Abschnitt f\"uhrt einige grundlegende graphentheroretische
Begriffe nach \cite{DBLP:conf/dagstuhl/BrandesE04a} ein.
  
Ein \emph{Netzwerk} bezeichnet eine Ansammlung von Objekten, zwischen
denen bilaterale Beziehungen bestehen. Das hier betrachtete Netzwerk
besteht aus einer Ansammlung von OpenPGP-Schl\"usseln, zwischen denen
Beziehungen in Form von Signaturen bestehen. Ein Netzwerk kann durch einen
Graphen repr\"asentiert werden.

Ein \emph{Graph} $G=(V, E)$ besteht aus einer endlichen Menge $V$ von
\emph{Knoten} und einer endlichen Menge $E$ von \emph{Kanten}, die je
zwei Knoten miteinander verbinden. Die Anzahl $|V$ der Knoten wird mit
$n$ und die Anzahl der Kanten mit $m$ bezeichnet. Zwei durch eine
Kante verbundene, \emph{benachbarte} Knoten heissen
\emph{adjazent}. In einem \emph{ungerichteten} Graphen ist $E$ eine
Teilmenge von $V\times V$, also eine Menge von Kanten $\{u, v\}$, die
zwei Knoten $u \in V$ und $v\in V$ verbinden. In einem
\emph{gerichteten} Graphen besteht eine Kante aus einem geordneten
Paar $(u, v)$ von Knoten. Eine gerichtete Kante $(u, v)$ f\"uhrt vom
\emph{Ursprung} $u$ zum \emph{Ziel} $v$.

Der \emph{Grad} $d(v)$ eines Knotens $v$ in einem ungerichteten
Graphen bezeichnet die Anzahl der Kanten, die diesen Knoten
enthalten. Die \emph{Nachbarschaft} $N(v)$ bezeichnet die Menge der
Knoten, die zu $v$ adjazent sind. In einem gerichteten Graphen
bezeichnet der \emph{eingehende Grad} $d^{-}(v)$ die Anzahl der
Kanten, die den Knoten $v$ als Ziel enthalten, und der
\emph{ausgehende Grad} $d^{+}(v)$ die Anzahl der Kanten, die $v$ als
Ursprung enthalten.

Ein \emph{Kantenzug} zwischen Knoten $v_0$ und $v_k$ in einem Graphen
$G=(V, E)$ ist eine alternierende Folge $(v_0, e_1, v_1, e_2, \dots,
v_{k-1}, e_k, v_k)$, wobei $v_i \in V$, $e_i \in E$, sowie $e_i =
\{v_{i-1}, v_{i}\}$ in einem ungerichteten Graphen bzw. $e_i =
(v_{i-1}, v_{i})$ in einem gerichteten Graphen. Der Kantenzug hat die
\emph{L\"ange} $i$. Ein Kantenzug heisst \emph{einfach}, wenn $e_i \ne
e_j$ f\"ur $i \ne j$ gilt.  Ein einfacher Kantenzug heisst
\emph{Pfad}, wenn $v_0, \dots, v_k$ paarweise verschieden sind. Die
\emph{Distanz} $d(u, v)$ von einem Knoten $u$ zu einem Knoten $v$ ist die
L\"ange eines k\"urzesten Pfades zwischen $u$ und $v$.

Ein Graph $G' = (V', E')$ ist ein \emph{Teilgraph} eines Graphen $G =
(V, E)$, wenn $E' \subseteq E$ und $V' \subseteq V$ gelten. Ein
Teilgraph $G' = (V', E')$ eines Graphen $G$ ist durch eine eine
Knotenmenge $V'$ \emph{induziert}, wenn $E' = \{(u, v) : u \in V, v
\in V, (u, v) \in E\}$ gilt.

Eine Partitionierung eines Graphen bezeichnet eine Zerlegung der
Knotenmenge in disjunkte Teilmengen.

Ein ungerichteter Graph heisst \emph{zusammenh\"angend}, wenn es
zwischen allen Knotenpaaren $u, v$ einen Pfad gibt. Eine
\emph{Zusammenhangskomponente} eines Graphen $G$ ist ein
\emph{maximaler}, zusammenh\"angender induzierter Teilgraph von
$G$. Ein gerichteter Graph ist \emph{stark zusammenh\"angend}, wenn es
f\"ur Knoten einen gerichteten Pfad zu jedem anderen Knoten gibt. Eine
\emph{starke Zusammenhangskomponente} eines Graphen $G$ ist dann ein
maximaler, stark zusammenh\"angender induzierter Teilgraph von $G$.

\subsection{Netzwerkstatistiken}
\label{ch:Grundlagen:sec:Netzwerkanalyse:subsec:Statistiken}

In diesem Abschnitt werden einige Kennzahlen nach
\cite{Brinkmeier2004} definiert, die die Struktur eines Graphen
charakterisieren.

Die Distanz zwischen zwei Knoten wurde bereits im vorherigen Abschnitt
definiert. Ausgehend davon k\"onnen die Distanzen \emph{eines Knotens}
durch die durchschnittliche Distanz dieses Knotens zu allen anderen
Knoten charakterisiert werden:

\begin{equation}
  \label{eq:1}
  \bar{d}(u) = \frac{1}{n-1} \sum_{v \ne u} d(u, v)
\end{equation}

F\"ur einen gesamten Graphen gibt die \emph{charakteristische oder
  durchschnittliche Distanz} den Durchschnitt aller Distanzen in
diesem Graphen an:

\begin{equation}
  \label{eq:2}
  \bar{d} = \frac{1}{n^2 - n} \sum_{u \ne v \in V} d(u, v)
\end{equation}

Die \emph{Eccentricity} eines Knotens $u$ ist definiert als die
maximale Distanz zu irgend einem anderen Knoten, also

\begin{equation}
  \label{eq:3}
  \epsilon(u) = \max\{d(u,v) | v \in V\}
\end{equation}

Davon ausgehend  wird der \emph{Durchmesser} eines Graphen als Maximum
und der Radius als Minimum \"uber die Eccentricity aller Knoten
definiert. Durchmesser und Radius geben also die Ober-
bzw. Untergrenze der Pfadl\"ange an, mit der ein Knoten alle anderen
Knoten erreichen kann. Bemerkt werden muss, dass alle auf Distanzen
basierenden Kennzahlen im Falle eines nicht (stark)
zusammenh\"angenden Graphen unendlich sind. Dieses Problem wird im
weiteren vermieden, indem diese Kennzahlen nur f\"ur (starke)
Zusammenhangskomponenten berechnet werden.

Als Verallgemeinerung der Nachbarschaft eines Knotens wird die
\emph{h-Nachbarschaft} eines Knotens als
\begin{equation}
  \label{eq:4}
  N_h(v) = \{ u \in V | d(v, u) \le h \}
\end{equation}
definiert, d.h. die Menge der Knoten, zu denen die Distanz von $v$ aus
h\"ochstens $h$ betr\"agt.


Assortativity Coefficient \cite{PhysRevLett.89.208701}.
\subsection{Netzwerkmodelle}
\label{sec:netzwerkmodelle}

\cite{Barabasi1999} 
baz

\subsection{Communities}
\label{ch:Grundlagen:sec:Netzwerkanalyse:subsec:Communities}

\subsubsection{Grundlegende Begriffe}
\label{sec:grundl-begr}

foo 

\subsubsection{Modularity}
\label{sec:modularity}

\begin{equation}
  \label{eq:modularity}
  Q =
  \frac{1}{2m}\sum_{ij}\left(A_{ij}-\frac{d_id_j}{2m}\right)\delta(c_i, c_j)
\end{equation}

\subsubsection{Algorithmen}
\label{sec:algorithmen}

[1. Algorithmus von Newman: Edge-Betweeness]

Seit der Arbeit von Newman wurde eine Vielzahl weiterer Methoden zur
Erkennung von Communities vorgestellt. An dieser Stelle werden 3
Methoden n\"aher beschrieben, die in dieser Arbeit verwendet
werden. Einen Gesamt\"uberblick bietet ein \"Ubersichtsartikel von
Fortunato \cite{Fortunato2010}.

%% ==============================
\section{Verwandte Arbeiten}
%% ==============================
\label{ch:Grundlagen:sec:RelatedWork}

\subsection{Analysen des OpenPGP-Web of Trust}
\label{ch:Grundlagen:sec:RelatedWork:subsec:wot-analysis}

\subsection{Community-Strukturen allgemein}
\label{ch:Grundlagen:sec:RelatedWork:subsec:community-analysis}
%%% Local Variables: 
%%% mode: latex
%%% TeX-master: "diplarb"
%%% End: 
