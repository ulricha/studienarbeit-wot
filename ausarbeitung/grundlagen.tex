%% grundlagen.tex
%% $Id: grundlagen.tex 28 2007-01-18 16:31:32Z bless $
%%

\chapter{Grundlagen}
\label{ch:Grundlagen}
%% ==============================
Die Grundlagen müssen soweit beschrieben
werden, dass ein Leser das Problem und
die Problemlösung  versteht.Um nicht zuviel 
zu beschreiben, kann man das auch erst gegen 
Ende der Arbeit schreiben.

Bla fasel\ldots

%% ==============================
\section{Kryptographie mit öffentlichen Schlüsseln}
%% ==============================
\label{ch:Grundlagen:sec:PublicKeyCrypto}

\subsection{Prinzip}
\label{ch:Grundlagen:sec:PublicKeyCrypto:subsec:Prinzip}

\subsection{Authentisierung von Schlüsseln}
\label{ch:Grundlagen:sec:PublicKeyCrypto:subsec:KeyAuth}

\subsubsection{Zentrale PKI}
\label{ch:Grundlagen:sec:PublicKeyCrypto:subsec:KeyAuth:subsubsec:PKI}

\subsubsection{Web of Trust}
\label{ch:Grundlagen:sec:PublicKeyCrypto:subsec:KeyAuth:subsubsec:WOT}

%% ==============================
\section{OpenPGP (RFC4880)}
%% ==============================
\label{ch:Grundlagen:sec:OpenPGP}

\subsection{Paketformat v4}
\label{ch:Grundlagen:sec:OpenPGP:subsec:PaketFormat}

\subsection{Unterschiede v3}
\label{ch:Grundlagen:sec:OpenPGP:subsec:v3Format}

\section{Graphentheorie allgemein}
\label{ch:Grundlagen:sec:Graphentheorie}

\section{Netzwerkanalyse}
\label{ch:Grundlagen:sec:Netzwerkanalyse}

\subsection{Netzwerkstatistiken}
\label{ch:Grundlagen:sec:Netzwerkanalyse:subsec:Statistiken}

\subsection{Communities}
\label{ch:Grundlagen:sec:Netzwerkanalyse:subsec:Communities}





%% ==============================
\section{Verwandte Arbeiten}
%% ==============================
\label{ch:Grundlagen:sec:RelatedWork}



\subsection{Analysen des OpenPGP-Web of Trust}
\label{ch:Grundlagen:sec:RelatedWork:subsec:wot-analysis}

\subsection{Community-Strukturen allgemein}
\label{ch:Grundlagen:sec:RelatedWork:subsec:community-analysis}



%%% Local Variables: 
%%% mode: latex
%%% TeX-master: "diplarb"
%%% End: 
