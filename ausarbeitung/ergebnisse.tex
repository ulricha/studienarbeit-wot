%% ergebnisse.tex

\chapter{Ergebnisse}
\label{ch:Ergebnisse}

\section{Allgemeine Merkmale des Netzwerkes}
\label{sec:allg-merkm-des}

\section{Struktur der starken Zusammenhangskomponenten}
\label{sec:komponentenstruktur}

\begin{figure}[h]
  \centering
  \includegraphics[scale=1.0]{images/component-metagraph-8.pdf}
  \caption{Struktur der starken Zusammenhangskomponenten bis zur
    Grösse 6 (rot = Grösse 6-10, grün = Grösse 11-20, blau = Grösse
    21-30, violett = Grösse $>= 31$, gelb = MSCC)}
  \label{fig:komponenten-struktur}
\end{figure}

\section{Zusammenhangskomponenten und Communities}
\label{sec:zusamm-und-comm}

\begin{figure}[h]
  \centering
  \includegraphics[scale=0.8]{images/community-size-dist.pdf}
  \caption{Gr\"ossenverteilung der von Fast-Modularity (FASTMOD),
    Label-Propagation (LABELPROP) und COPRA berechneten Communities}
  \label{fig:community-size-dist}
\end{figure}

\begin{table}[h]
  \centering
  \begin{tabular}{l|c|p{4cm}|p{4cm}}
    Algorithmus & Modularity & Anzahl Communities (Gr\"osse $> 3$) &
    Anzahl enthaltener Knoten \\
    \hline
    fastmod & 0,596 & 552 & foo \\
    labelprop & 0,658 & 1834 & foo \\
    cliquemod-3000 & 0,670 & 811 & foo \\
    cliquemod-500 & 0,677 & 451 & foo \\
    copra & --- & 1354 & foo
  \end{tabular}
  \caption{Modulairty, Anzahl communities, Menge enthaltener Knoten (> 3)}
  \label{tab:foo}
\end{table}

{\footnotesize
\begin{table}[h]
  \centering
  \begin{tabular}{l|c|c|c|c|c|c}
    Algo & TLD-S & TLD-M & SLD-S & SLD-M & SIG \\
    \hline
    fastmod & 293 (53\%) & 247 (44\%) & 66 (11\%) & 194 (35\%) & 128
    (23\%) \\
    \hline
    labelprop & 1048 (57\%) & 755 (41\%) & 277 (15\%) & 720 (39\%) &
    553 (30\%) \\
    \hline
    cliquemod-3000 & 460 (56\%) & 332 (40\%) & 104 (12\%) & 300 (36\%)
    & 230 (28\%) \\
    \hline
    cliquemod-500 & 230 (50\%) & 209 (46\%) & 65 (14\%) & 141 (31\%) &
    116 (25\%) \\
    \hline
    copra & 792 (58\%) & 525 (38\%) & 259 (19\%) & 533 (39\%) & 555 (40\%)

  \end{tabular}
  \caption{SLD-TLD-Zuweisung, TIME-CORR}
  \label{tab:assign}
\end{table}
}

\begin{figure}[h]
  \centering
  \includegraphics[scale=1.7]{images/fastmod-subgraph-large-modular-6525064ccab580a0b304a3620b197d7a.pdf}
  \caption{Grosse Community mit modularer Struktur (fastmod,
    force-directed layout)}
  \label{fig:large-community-modular}
\end{figure}
\begin{figure}[h]
  \centering
  \includegraphics[scale=1.5]{images/label-prop-metagraph-20-narrow-cut.pdf}
  \caption{Struktur der Communities (Label-Propagation) (rot: Gr\"osse
    $<50$, gr\"un: Gr\"osse $<100$, blau: Gr\"osse < 1000, gelb:
    Gr\"osse > 1000).}
  \label{fig:metagraph-com-label}
\end{figure}




Modularity ist auf ung
\begin{figure}[h]
  \centering
  \includegraphics[scale=1.0]{images/fastmod-metagraph-20.pdf}
  \caption{Struktu der Communities (Fast Modularity) (Farbkodierung
    wie in Abb.\ref{fig:metagraph-com-label})}
  \label{fig:metagraph-com-fastmod}
\end{figure}

\begin{figure}[h]
  \centering
  \includegraphics[scale=1.0]{images/subgraph-label-time-1077361cfa4161f22a2e4abb5ca89b8c1ad.pdf}
  \caption{foo}
  \label{fig:time-corr-com-normal}
\end{figure}

\begin{figure}[h]
  \centering
  \includegraphics[scale=1.0]{images/label-subgraph-41-star-c222345bc5eb1f9eff80d58a81861974.pdf}
  \caption{foo}
  \label{fig:time-corr-com-star}
\end{figure}

\begin{figure}[h]
  \centering
  \includegraphics[scale=0.8]{images/tld_sure_dist.pdf}
  \caption{Zuweisung von Domains zu TLDs abh\"angig von der Community-Gr\"osse}
  \label{fig:tld-sure-ass-dist}
\end{figure}

\begin{figure}[h]
  \centering
  \includegraphics[scale=0.8]{images/tld_maybe_dist.pdf}
  \caption{Verteilung der Gr\"osse der von einer TLD dominierten Communities}
  \label{fig:tld-maybe-ass-dist}
\end{figure}

\begin{figure}[h]
  \centering
  \includegraphics[scale=0.8]{images/sld_sure_dist.pdf}
  \caption{foo}
  \caption{Zuweisung von Domains zu SLDs abh\"angig von der Community-Gr\"osse}
  \label{fig:sld-sure-ass-dist}
\end{figure}

\begin{figure}[h]
  \centering
  \includegraphics[scale=0.8]{images/sld_maybe_dist.pdf}
  \caption{Verteilung der Gr\"osse der von einer TLD dominierten Communities}
  \label{fig:sld-maybe-ass-dist}
\end{figure}

\begin{figure}[h]
  \centering
  \includegraphics[scale=0.8]{images/time_corr_dist.pdf}
  \caption{Verteilung der Gr\"osse der Communities mit zeitlicher Korrelation}
  \label{fig:time-corr-dist}
\end{figure}


\subsection{Communities in gerichteten Graphen}
\label{sec:comm-gericht-graph}
erichteten Graphen definiert. Ebenso 

%%% Local Variables: 
%%% mode: latex
%%% TeX-master: "diplarb"
%%% End: 
