%% methoden.tex

\chapter{Methoden und Materialien FIXME}
\label{ch:Methoden}
%% ==============================

%% ==============================
\section{Warum eigene Software?}
%% ==============================
\label{ch:Grundlagen:sec:WarumEigene}
Das bereits in Abschnitt \ref{ch:Grundlagen:sec:RelatedWork} erwähnte
\emph{wotsap}-Projekt berechnet täglich die Struktur des Web of
Trust. Die Daten werden für eine Web-Applikation benutzt, die
grundlegende Statistiken über Schlüssel berechnet und Pfade zwischen
Schlüsseln graphisch darstellt. Zusätzlich werden die Daten aber auch
für weitere Analysen zur Verfügung gestellt. In diesem Abschnitt wird
begründet, warum für die vorliegende Arbeit nicht auf diese Daten
zurückgegriffen wurde, sondern die Datenextraktion selbst vorgenommen
wurde.

Ein Ziel dieser Arbeit ist es, die Struktur des Web of Trust abseits
der grössten starken Zusammenhangskomponente zu
untersuchen. \emph{wotsap} berechnet allerdings nur die Struktur eben
dieser Zusammenhangskomponente. Schlüssel, die nicht in dieser
Komponente enthalten sind, werden nicht beachtet. \emph{wotsap}
beginnt bei mehreren sehr gut vernetzten Schlüsseln, die sicher in der
MSCC liegen. Von diesen ausgehend werden die Signaturen in der Art
einer Breitensuche (rückwärts) verfolgt. Aufgrund dieser Methode
scheint es mit vertretbarem Aufwand nicht möglich, die Extraktion auf
alle Schlüssel auszudehnen.

Der Anwendungszweck der \emph{wotsap}-Daten ist ausschliesslich die
strukturelle Analyses des Netzes. Die über die reine Struktur
hinausgehenden Daten, die über Schlüssel und Signaturen gespeichert
werden, sind dabei auf ein Minimum reduziert: Für Schlüssel werden
ausschliesslich die KeyID und die primäre UserID gespeichert, für
Signaturen der Cert level und die Schlüssel. Diese Reduktion erlaubt
zwar eine sehr kompakte Speicherung der Daten, macht es aber für eine
Auswertung weiterer Eigenschaften von Schlüsseln und Signaturen
unbrauchbar. Das verwendete Dateiformat ist ausserdem recht unflexibel
und lässt eine Speicherung weiterer Daten nur mit grösserem Aufwand
zu.

Die \emph{wotsap}-Daten beinhalten nur die zum jeweiligen Zeitpunkt
gültigen Schlüssel und Signaturen. Dieser "`Schnappschuss"' reicht für
die strukturelle Analyse des Graphen aus. Zeitliche Entwicklungen,
beispielsweise die Grösse des Datenbestandes, die Verwendung
bestimmter Verschlüsselungs- und Signaturalgorithmen und die Entwicklung
einzelner Komponenten können damit aber nicht nachvollzogen werden.

Die \emph{wotsap}-Methode liefert also nicht die im Rahmen dieser
Arbeit benötigten Daten. Eine Anpassung der Software würde auf eine
komplette Neuimplementierung hinauslaufen. Ausserdem beruht die
Extraktion der Daten bei \emph{wotsap} auf der veralteten und kaum
mehr benutzten \emph{PKS}-Keyserver-Implementierung (siehe Abschnitt
\ref{ch:Grundlagen:sec:Design:subsec:der-sks-keyserver}).

Darüber hinaus ist allerdings der \emph{wotsap}-Datensatz
fehlerhaft. Diese Fehler werden sowohl durch Fehler in der
Implementierung als auch durch einen fehlerhaften Datenbestand auf dem
verwendeten Keyserver verursacht: Die grösste starke
Zusammenhangskomponente die durch XXX berechnet %FIXME
wurde (MSCC-1), enthält mit dem Stand vom 02.12.2009 ca. 45100
Schlüssel, während der \emph{wotsap}-Datensatz (MSCC-2) vom gleichen
Tag nur ca. 42130 Schlüssel enthält. Die Differenz zwischen den
Datensätzen ergibt einerseits, dass ca. 1000 Schlüssel in MSCC-1
fehlen, die in MSCC-2 vorhanden sind. Eine stichprobenartige Analyse
von 20 Schlüsseln zeigt, dass diese Schlüssel überwiegend durch
Signaturketten an die MSCC angebunden sind, die aufgrund von
widerrufenen Schlüsseln oder Schlüsselteilen unterbrochen sind. Dabei
treten u.a. Signaturen auf komplett widerrufenen Schlüsseln und
Signaturen auf widerrufenen UserIDs auf. \emph{wotsap} benutzt GnuPG,
um die OpenPGP-Pakete eines Schlüssels zu parsen. Der Code zum Parsen
der GnuPG-Ausgabe ist fehlerhaft und führt dazu, dass \emph{wotsap}
Widerrufssignaturen fäschlicherweise nicht beachtet.

Auf der anderen Seite sind ca. 3980 Schlüssel zwar in MSCC-1
vorhanden, nicht aber in MSCC-2. Um den Grund dafür zu finden, wurde
für 10 zufällig ausgewählte Schlüssel aus dieser Menge eine
Signaturkette (d.h. ein Pfad) zu einem Schlüssel gesucht, der sowohl
in MSCC-1 als auch in MSCC-2 vorhanden ist. Ebenfalls wurde eine Kette
in der anderen Richtung gesucht. Die Signaturen dieser Ketten wurden
mittels GnuPG kryptographisch verifiziert, um sicherzustellen, dass
sie gültig sind. Kann für beide Richtungen jeweils eine Kette
erfolgreich verifiziert werden, so ist der betreffende Schlüssel per
definitionem in der MSCC. Im vorliegenden Fall konnte das für alle
Schlüssel der Stichprobe gezeigt werden. Der Fehler liegt also
wiederrum bei \emph{wotsap}, dass diese Schlüssel fälschlicherweise
ausschliesst. Als Ursache dafür ergab sich in allen betrachteten
Fällen der fehlerhafte Bestand des Keyservers
\emph{wwwkeys.ch.pgp.net}, der von \emph{wotsap} verwendet wird. Die
Schlüssel wurden von \emph{wotsap} nicht in die MSCC-2 übernommen,
weil Teile der dazu notwendigen Signaturketten (komplette Schlüssel
oder einzelne Signaturen) auf dem Keyserver nicht vorhanden
sind. Diese fehlenden Teile sind aber auf allen Keyservern des
\emph{SKS}-Verbundes %FIXME
vorhanden. Als Ursache kommen eine mangelhafte Synchronisation
zwischen \emph{wwwkeys.ch.pgp.net} und dem \emph{SKS}-Netzwerk sowie
Fehler in der auf \emph{wwwkeys.ch.pgp.net} verwendeten
\emph{PKS}-Version in Frage.

Die Natur der Fehler legt nahe, dass hauptsächlich solche Schlüssel
fehlen bzw. fälschlich einbezogen wurden, die nur über eine sehr
geringe Anzahl von redundanten Pfaden zur bzw. von der MSCC
verfügen. Gäbe es mehr redundante Pfade, so hätten einzelne
fehlerhafte Informationen (fehlende bzw. fäschlicherweise als gültig
betrachtete Schlüssel oder Signaturen) einen geringeren Einfluss.

Die Anzahl fehlender bzw. fälschlich einbezoger Schlüssel (ca. 9\%
fehlend, ca. 2\% fälschlich einbezogen relativ zu MSCC-1) ist so
gross, dass qualitative Fehler in der Graphenstruktur zu erwarten
sind. Insbesondere Aussagen über die Verteilung von Knotengraden und
ähnliche Aussagen anhand dieser Daten sind mit Vorsicht zu
geniessen. Eine Reihe von Arbeiten verwendet die \emph{wotsap}-Daten
als Beispiel für ein empirisches soziales bzw. Small-World-Netzwerk
\cite{Brondsema2006} \cite{Heikkila2009} \cite{Dell'Amico2007}. Sofern
die Ergebnisse dieser Arbeit auf experimentellen Resultaten aufbauen,
die mit diesen Daten gewonnen wurden, sollten sie daher mit korrekten
Daten überprüft werden.



\section{Design und Implementierung}
\label{ch:Grundlagen:sec:Design}

\subsection{Der SKS Keyserver}
\label{ch:Grundlagen:sec:Design:subsec:der-sks-keyserver}




%%% Local Variables: 
%%% mode: latex
%%% TeX-master: "diplarb"
%%% End: 
