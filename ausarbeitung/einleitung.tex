%% Einleitung.tex
%% $Id: einleitung.tex 28 2007-01-18 16:31:32Z bless $
%%

\chapter{Einleitung}
\label{ch:Einleitung}
%% ==============================

Die meisten Kommunikationsmedien -- insbesondere E-Mail -- können
nicht garantieren, dass Nachrichten unterwegs abgehört, verändert
oder mit gefälschtem Absender verschickt werden. PGP und GnuPG sind
Softwarepakete, die -- basierend auf asymmetrischer Kryptographie --
Vertraulichkeit und Authentizität gewährleisten.

Asymmetrische Kryptographie erlaubt es, Schlüssel auf unsicheren
Kanälen auszutauschen. Damit stellt sich aber ein neues Problem: Wie
kann ein Benutzer sicherstellen, dass ein Schlüssel authentisch ist,
also tatsächlich vom angeblichen Besitzer stammt? Im Kontext
elektronischer Kommunikation kann dies meist nicht durch direkten
Kontakt mit dem Kommunikationspartner festgestellt werden. Statt
dessen müssen sich Teilnehmer auf Dritte verlassen, die Schlüssel
verifizieren.

Der traditionelle Ansatz dafür besteht in der Verwendung von
zentralen und hierarchischen Infrastrukturen, die sich auf wenige
zentrale Autoritäten verlassen. Diese Autoritäten nehmen die
Aufgabe der Verifizierung von Schlüsseln wahr.  PGP setzt
stattdessen auf das Paradigma der Selbstorganisierung in Form des
\emph{Web of Trust}. Mit Selbstorganisierung ist gemeint, dass ein
System nicht zentral geplant wird, sondern sich ausschließlich aus
den Interaktionen aller Teilnehmer ergibt. Im Web of Trust ist die
Aufgabe der Verifizierung von Schlüsseln auf \emph{alle} Teilnehmer
verlagert. Diese stellen Informationen über von ihnen verifizierte
Schlüssel öffentlich zur Verfügung und ermöglichen es so
anderen Teilnehmern, sie zu benutzen. Auf diese Weise ergibt sich ein
Netzwerk, das abbildet, welche Schlüssel (bzw. Benutzer) welche
anderen Schlüssel verifiziert haben.

Eine Vielzahl von Arbeiten befasst sich mit der Analyse der Struktur
großer Netzwerke, die in verschiedensten Bereichen angetroffen
werden. Dies können etwa soziale, technische oder biologische
Netzwerke sein. In den letzten Jahren hat insbesondere die Analyse von
sozialen und technischen Netzwerken durch die Verfügbarkeit von
Datensammlungen aus dem Bereich des Internets, von Online Social
Networks (Facebook etc.) u.ä. einen erheblichen Auftrieb erhalten.

Das Web of Trust stellt ein Beispiel für ein solches Netzwerk dar,
das sich aus der Selbstorganisierung der Teilnehmer ergibt. Auch wenn
es zunächst nicht über die streng regelmäßige Struktur eines
geplanten Systems verfügt, ist es auch nicht komplett
unstrukturiert. Es ergeben sich Strukturen, die die Beziehungen der
Teilnehmer im realen Leben widerspiegeln. Das Netzwerk kann
einerseits nur in Bezug auf die Erfüllung seines Zwecks betrachtet
werden: Wie stark wird das Web of Trust benutzt und wie gut erfüllt
es seinen Zweck? Andererseits kann seine Struktur benutzt werden, um
die Mechanismen seiner Entstehung zu erklären. Wie finden sich die
Mechanismen seiner Entstehung und die abgebildeten Beziehungen in
dieser Struktur wieder? Diese Studienarbeit soll zur Beantwortung beider Fragen beitragen.


%% ==============================
\section{Zielsetzung der Arbeit}
%% ==============================
\label{ch:Einleitung:sec:Zielsetzung}

Das Ziel dieser Studienarbeit ist eine Analyse der Struktur des Web of
Trust auf verschiedenen Ebenen. Dazu soll zunächst eine Software
implementiert werden, die das Web of Trust aus einer
Schlüsseldatenbank extrahiert und in Form eines Graphen verfügbar
macht. 

Der dadurch erhaltene Graph soll unter mehreren Gesichtspunkten
analysiert werden:

\begin{itemize}

\item Wie stark sind die Teilnehmer des Web of Trust vernetzt?

\item Bekannt ist aus der Literatur, dass sich im Web of Trust eine starke
  Zusammenhangskomponente findet, die deutlich gr\"o{\ss}er ist als
  die restlichen Komponenten. Allerdings enth\"alt diese trotzdem nur
  einen kleinen Teil der insgesamt vorhandenen Schl\"ussel. Hier soll
  auch die Struktur des restlichen Netzwerks betrachtet werden.

\item  Wie nützlich ist das Web of Trust für seine Teilnehmer? Als
  Mass für die Nützlichkeit wird hier angesehen, wie viele
  Schlüssel für Teilnehmer anhand der Signaturen im Netzwerk
  prinzipiell verifizierbar sind.

\item Wie robust ist das Web of Trust? Wie sehr hängt sein
  Zusammenhang von einzelnen Schlüsseln ab?

\item Wie häufig werden verschiedene Verschlüsselungs- und
  Hashalgorithmen bei PGP-Schlüsseln verwendet? Welche Auswirkungen
  haben Fortschritte in Angriffen gegen diese Algorithmen?

\item Spiegeln sich Gruppierungen von Individuen in der Struktur des
  Web of Trust wieder? Lässt sich anhand solcher Gruppen erklären,
  wie sich das Netzwerk bildet?
\end{itemize}

%% ==============================
\section{Gliederung der Arbeit}
%% ==============================
\label{ch:Einleitung:sec:Gliederung}

Die Arbeit ist im weiteren wie folgt gegliedert: Kapitel
\ref{ch:Grundlagen} erläutert die notwendigen Grundlagen. Dies sind
zum einen die Prinzipien asymmetrischer Kryptographie (Abschnitt
\ref{ch:Grundlagen:sec:PublicKeyCrypto}), Methoden zur
Authentifizierung öffentlicher Schlüssel, die Weiterhin werden in
Abschnitt \ref{ch:Grundlagen:sec:PGP} die Prinzipien von PGP und GnuPG
sowie die Rahmenbedingungen, in denen PGP/GnuPGP verwendet wird,
beschrieben. Außerdem werden in Abschnitt \ref{sec:graph-und-netzw}
die graphentheoretischen und netzwerkanalytischen Verfahren
eingeführt, die in dieser Arbeit verwendet werden. Abschnitt
\ref{ch:Grundlagen:sec:RelatedWork} bespricht verwandte Arbeiten.

Kapitel \ref{ch:Methoden} beschreibt die verwendeten
Methoden. Abschnitt \ref{ch:Grundlagen:sec:WarumEigene} begründet
zunächst, warum die Extraktion der benötigen Daten selbst
implementiert wurde. Abschnitt \ref{ch:Grundlagen:sec:Design} beschreibt die
im Rahmen dieser Arbeit implementierte Software zur Extraktion und
Analyse des Web of Trust, während Abschnitt
\ref{sec:community-analyse} die Vorgehensweise für die Analyse der
Community-Struktur in Abschnitt \ref{sec:result-zusamm-und-comm}
erläutert.

Kapitel \ref{ch:Ergebnisse} stellt die Ergebnisse vor und
diskutiert sie. Dies gliedert sich in Ergebnisse über die Struktur
des Web of Trust (Abschnitt \ref{sec:result-allg-merkm-des}), einige
Statistiken über Eigenschaften einzelner Schlüssel und die
zeitliche Entwicklung des Web of Trust (Abschnitt
\ref{sec:result-key-properties}) sowie die Analyse der
Community-Struktur (Abschnitt \ref{sec:result-zusamm-und-comm}).

Kapitel \ref{ch:Zusammenfassung} schließlich fasst die Ergebnisse
zusammen und bietet einen Ausblick auf Ansatzpunkte für zukünftige
Arbeiten.

%%% Local Variables: 
%%% mode: latex
%%% TeX-master: "diplarb"
%%% End: 
