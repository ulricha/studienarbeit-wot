%% Einleitung.tex
%% $Id: einleitung.tex 28 2007-01-18 16:31:32Z bless $
%%

\chapter{Einleitung}
\label{ch:Einleitung}
%% ==============================
Hinweis: In die Einleitung gehört die Motivation 
und Einleitung in die Problemstellung. Die Problemstellung
kann in der Analyse noch detaillierter beschrieben werden.

Bla fasel\ldots
foo


%% ==============================
\section{Zielsetzung der Arbeit}
%% ==============================
\label{ch:Einleitung:sec:Zielsetzung}

Arbeit befasst sich nicht mit den zugrundeliegenden kryptographischen
Verfahren, sondern mit denen Rahmenbedinungen, in denen diese bei
GnuPG verwendet werden.

Was ist die Aufgabe der Arbeit?

Bla fasel\ldots

%% ==============================
\section{Gliederung der Arbeit}
%% ==============================
\label{ch:Einleitung:sec:Gliederung}

Die Arbeit ist im weiteren wie folgt gegliedert: Kapitel
\ref{ch:Grundlagen} erl\"autert die notwendigen Grundlagen. Dies sind
zum einen die Prinzipien asymmetrischer Kryptographie (Abschnitt
\ref{ch:Grundlagen:sec:PublicKeyCrypto}), Methoden zur
Authentifizierung \"offentlicher Schl\"ussel, die Weiterhin werden in
Abschnitt \ref{ch:Grundlagen:sec:PGP} die Prinzipien von PGP und GnuPG
sowie die Rahmenbedingungen, in denen PGP/GnuPGP verwendet wird,
beschrieben. Ausserdem werden in \ref{sec:graph-und-netzw} die
graphentheoretischen und netzwerkanalytischen Verfahren eingef\"uhrt,
die in dieser Arbeit verwendet werden. Abschnitt
\ref{ch:Grundlagen:sec:RelatedWork} bespricht verwandte Arbeiten.

Kapitel \ref{ch:Methoden} beschreibt die verwendeten
Methoden. Abschnitt \ref{ch:Grundlagen:sec:WarumEigene} begr\"undet
zun\"achst, warum die Extraktion der ben\"otigen Daten selbst
implementiert wurde. Abschnitt \ref{ch:Grundlagen:sec:Design} beschreibt die
im Rahmen dieser Arbeit implementierte Software zur Extraktion und
Analyse des Web of Trust, w\"ahrend Abschnitt
\ref{sec:community-analyse} die Vorgehensweise f\"ur die Analyse der
Community-Struktur in Abschnitt \ref{sec:result-zusamm-und-comm}
erl\"autert.

Kapitel \ref{ch:Ergebnisse} stellt die erhaltenen Ergebnisse vor und
diskutiert sie. Dies gliedert sich in Ergebnisse \"uber die Struktur
des Web of Trust (Abschnitt \ref{sec:result-allg-merkm-des}), einige
Statistiken \"uber Eigenschaften einzelner Schl\"ussel und die
zeitliche Entwicklung des Web of Trust (Abschnitt
\ref{sec:result-key-properties}) sowie die Analyse der
Community-Struktur (Abschnitt \ref{sec:result-zusamm-und-comm}).

Kapitel \ref{ch:Zusammenfassung} schliesslich fasst die Ergebnisse
zusammen und bietet einen Ausblick auf Ansatzpunkte f\"ur zuk\"unftige
Arbeiten.

%%% Local Variables: 
%%% mode: latex
%%% TeX-master: "diplarb"
%%% End: 
