%% Einleitung.tex
%% $Id: einleitung.tex 28 2007-01-18 16:31:32Z bless $
%%

\chapter{Einleitung}
\label{ch:Einleitung}
%% ==============================

Die meisten Kommunikationsmedien -- insbesondere E-Mail -- k\"onnen
nicht garantieren, dass Nachrichten unterwegs abgeh\"ort, ver\"andert
oder mit gef\"alschtem Absender verschickt werden. PGP und GnuPG sind
Softwarepakete, die -- basierend auf asymmetrischer Kryptographie --
Vertraulichkeit und Authentizit\"at gew\"ahrleisten.

Asymmetrische Kryptographie erlaubt es, Schl\"ussel auf unsicheren
Kan\"alen auszutauschen. Damit stellt sich aber ein neues Problem: Wie
kann ein Benutzer sicherstellen, dass ein Schl\"ussel authentisch ist,
also tats\"achlich vom angeblichen Besitzer stammt? Im Kontext
elektronischer Kommunikation kann dies meist nicht durch direkten
Kontakt mit dem Kommunikationspartner festgestellt werden. Statt
dessen m\"ussen sich Teilnehmer auf Dritte verlassen, die Schl\"ussel
verifizieren.

Der traditionelle Ansatz daf\"ur besteht in der Verwendung von
zentralen und hierarchischen Infrastrukturen, die sich auf wenige
zentrale Authorit\"aten verlassen. Diese Authorit\"aten nehmen die
Aufgabe der Verifizierung von Schl\"usseln wahr.  PGP setzt
stattdessen auf das Paradigma der Selbstorganisierung in Form des
\emph{Web of Trust}. Mit Selbstorganisierung ist gemeint, dass ein
System nicht zentral geplant wird, sondern sich ausschliesslich aus
den Interaktionen aller Teilnehmer ergibt. Im Web of Trust ist die
Aufgabe der Verifizierung von Schl\"usseln auf \emph{alle} Teilnehmer
verlagert. Diese stellen Informationen \"uber von ihnen verifizierte
Schl\"ussel \"offentlich zur Verf\"ugung und erm\"oglichen es so
anderen Teilnehmern, sie zu benutzen. Auf diese Weise ergibt sich ein
Netzwerk, dass abbildet, welche Schl\"ussel (bzw. Benutzer) welche
anderen Schl\"ussel verifizert haben.

Eine Vielzahl von Arbeiten befasst sich mit der Analyse der Struktur
grosser Netzwerke, die in verschiedensten Bereichen angetroffen
werden. Dies k\"onnen etwa soziale, technologische oder biologische
Netzwerke sein. In den letzten Jahren hat insbesondere die Analyse von
sozialen und technologischen Netzwerken durch die Verf\"ugbarkeit von
Datensammlungen aus dem Bereich des Internets, von Online Social
Networks (Facebook etc.) u.\"a. einen erheblichen Auftrieb erhalten.

Das Web of Trust stellt ein Beispiel f\"ur ein solches Netzwerk dar,
dass sich aus der Selbstorganisierung der Teilnehmer ergibt. Auch wenn
es zun\"achst nicht \"uber die streng regelm\"assige Struktur eines
geplanten Systems verf\"ugt, ist es auch nicht komplett
unstrukturiert. Es ergeben sich Strukturen, die die Beziehungen der
Teilnehmer im realen Leben wiederspiegeln. Das Netzwerk kann
einerseits nur in Bezug auf die Erf\"ullung seines Zwecks betrachtet
werden: Wie stark wird das Web of Trust benutzt und wie gut erf\"ullt
es seinen Zweck? Andererseits kann seine Struktur benutzt werden, um
die Mechanismen seiner Entstehung zu erkl\"aren. Wie finden sich die
Mechanismen seiner Entstehung und die abgebildeten Beziehungen in
dieser Struktur wieder? Diese Studienarbeit stellt einen Versuch dar,
zur Beantwortung beider Fragen beizutragen.


%% ==============================
\section{Zielsetzung der Arbeit}
%% ==============================
\label{ch:Einleitung:sec:Zielsetzung}

Das Ziel dieser Studienarbeit ist eine Analyse der Struktur des Web of
Trust auf verschiedenen Ebenen. Dazu soll zun\"achst eine Software
implementiert werden, die das Web of Trust aus einer
Schl\"usseldatenbank extrahiert und in Form eines Graphen verf\"ugbar
macht. 

Der dadurch erhaltene Graph soll unter mehreren Gesichtspunkten
analysiert werden:

\begin{itemize}

\item Wie stark sind die Teilnehmer des Web of Trust vernetzt?

\item  Wie n\"utzlich ist das Web of Trust f\"ur seine Teilnehmer? Als
  Mass f\"ur die N\"utzlichkeit wird hier angesehen, wie viele
  Schl\"ussel f\"ur Teilnehmer anhand der Signaturen im Netzwerk
  prinizpiell verifizierbar sind.

\item Wie robust ist das Web of Trust? Wie sehr h\"angt sein
  Zusammenhang von einzelnen Schl\"usseln ab?

\item Wie h\"aufig werden verschiedene Verschl\"usselungs- und
  Hashalgorithmen bei PGP-Schl\"usseln verwendet? Welche Auswirkungen
  haben Fortschritte in Angriffen gegen diese Algorithmen?

\item Spiegeln sich Gruppierungen von Individuen in der Struktur des
  Web of Trust wieder? L\"asst sich anhand solcher Gruppen erkl\"aren,
  wie sich das Netzwerk bildet?
\end{itemize}

%% ==============================
\section{Gliederung der Arbeit}
%% ==============================
\label{ch:Einleitung:sec:Gliederung}

Die Arbeit ist im weiteren wie folgt gegliedert: Kapitel
\ref{ch:Grundlagen} erl\"autert die notwendigen Grundlagen. Dies sind
zum einen die Prinzipien asymmetrischer Kryptographie (Abschnitt
\ref{ch:Grundlagen:sec:PublicKeyCrypto}), Methoden zur
Authentifizierung \"offentlicher Schl\"ussel, die Weiterhin werden in
Abschnitt \ref{ch:Grundlagen:sec:PGP} die Prinzipien von PGP und GnuPG
sowie die Rahmenbedingungen, in denen PGP/GnuPGP verwendet wird,
beschrieben. Ausserdem werden in Abschnitt \ref{sec:graph-und-netzw}
die graphentheoretischen und netzwerkanalytischen Verfahren
eingef\"uhrt, die in dieser Arbeit verwendet werden. Abschnitt
\ref{ch:Grundlagen:sec:RelatedWork} bespricht verwandte Arbeiten.

Kapitel \ref{ch:Methoden} beschreibt die verwendeten
Methoden. Abschnitt \ref{ch:Grundlagen:sec:WarumEigene} begr\"undet
zun\"achst, warum die Extraktion der ben\"otigen Daten selbst
implementiert wurde. Abschnitt \ref{ch:Grundlagen:sec:Design} beschreibt die
im Rahmen dieser Arbeit implementierte Software zur Extraktion und
Analyse des Web of Trust, w\"ahrend Abschnitt
\ref{sec:community-analyse} die Vorgehensweise f\"ur die Analyse der
Community-Struktur in Abschnitt \ref{sec:result-zusamm-und-comm}
erl\"autert.

Kapitel \ref{ch:Ergebnisse} stellt die erhaltenen Ergebnisse vor und
diskutiert sie. Dies gliedert sich in Ergebnisse \"uber die Struktur
des Web of Trust (Abschnitt \ref{sec:result-allg-merkm-des}), einige
Statistiken \"uber Eigenschaften einzelner Schl\"ussel und die
zeitliche Entwicklung des Web of Trust (Abschnitt
\ref{sec:result-key-properties}) sowie die Analyse der
Community-Struktur (Abschnitt \ref{sec:result-zusamm-und-comm}).

Kapitel \ref{ch:Zusammenfassung} schliesslich fasst die Ergebnisse
zusammen und bietet einen Ausblick auf Ansatzpunkte f\"ur zuk\"unftige
Arbeiten.

%%% Local Variables: 
%%% mode: latex
%%% TeX-master: "diplarb"
%%% End: 
